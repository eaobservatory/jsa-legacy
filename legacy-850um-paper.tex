\documentclass[usenatbib]{mn2e}

\usepackage{graphicx}
\usepackage{marginnote}
\usepackage{xcolor}

\newcommand{\aspconf}{ASP Conf. Ser.}
\newcommand{\todo}[1]{\textcolor{red}{TODO: #1}}
\title{The JCMT Science Archive: 850\micron Legacy Release}
\author[S.~F.~Graves et~al.]
{Sarah~F.~Graves,$^{1,2}$
JAC
and
Others.\\
$^1$Joint Astronomy Centre, 660 N.\ A`oh\=ok\=u Place, Hilo, HI 96720, USA\\
$^2$East Asian Observatory, 660 N.\ A`oh\=ok\=u Place, Hilo, HI 96720, USA
}


\begin{document}

\date{in development}

\pagerange{\pageref{firstpage}--\pageref{lastpage}} \pubyear{2015}

\maketitle

\label{firstpage}

\begin{abstract}

  The JCMT Science archive allows access to raw and reduced data for
  all publicly available JCMT observations. To aid users of JCMT
  products, we have produced a standardised reduction of public JCMT
  SCUBA-2 850\micron SCUBA-2 observations. These reductions provide
  uniformly reduced coadds of of all data that was public as of August
  1st 2014, and catalogues to identify the regions where emission was
  detected. The data have been gridded in the HEALPix HPX scheme.
\end{abstract}

\section{Introduction}
\begin{itemize}
\item Description of the JSA \citep{2015Economou}
\item Very brief description of heterodyne and SCUBA-2 instruments, cite instrument papers. \citep{2013MNRAS.430.2513H} \citep{2009MNRAS.399.1026B}
\item Outline of the advanced data products. \citep{2014SPIE.9152E..2JB}
\item Historical perspective. \citep{2011ASPC..442..203E}
\item ongoing plans -- dynamic archive, continually updated etc.
\end{itemize}



\section{HEALPix grid }
Require a fixed tiling and grid to reduce our data onto, as we want to be able to
incorporate all data current and future into this method. The chosen
grid is that of the HEALPix method (Hierarchical Equal Area
isoLatitude pixelation), commonly used in cosmological fields. This
has the advantage that all pixels have the same area. The trade off is
that the pixels do not have the same size in RA and Dec.

\todo{Insert Brief Description of healpix, advantages and
  disadvantages}

\todo{Create plots illustrating healpix}.

\todo{list parameters we have used}.

\footnote{To re-grid a HEALPix tile onto a standard RA-Dec projection,
the starlink software command 'XXXXXX' can be used.}

\citep{2005ApJ...622..759G}

\citep{2007MNRAS.381..865C}

\section{Grouping, reducing and coadding observations}
Describe the basic flow chart (and illustrate) for JSA advanced data
products (obs product reductions, grouping, coadd, cataloges, and
similar but with some co-reductions and frequency regridding for
HARP).

Full description of how the observations are grouped into
associations.



\subsection{SCUBA-2 reduction configuration}

\citep{2013MNRAS.430.2513H}

\begin{itemize}
\item Brief description of how SCUBA-2 data reduction works.
\item mapmaker paper: \citep{2013MNRAS.430.2545C}
\item Details of our chosen configuration and why we chose it. (detailed).
\item Examples of some reduced maps.
\end{itemize}


\subsection{QA}
Description of process. Examples of observations we threw out.


\subsection{Co-adding}
Brief?
\begin{itemize}
\item Do we include anything to cope with hot pixels etc.
\end{itemize}



\section{Calibration}
\begin{itemize}
\item Calibration paper \citep{2013MNRAS.430.2534D}
\item show images the standard calibrators.
\item comparison with 'standard' calibrator reductions
\item effect of pixel size on fluxes.
\item effect of pointing errors on fluxes.
\item Our uncertainty in the final flux.
\end{itemize}


We calibrated our data into units of mJy per arc-second squared.
As the effective beam size is increased during the co-add procedure
(due to errors in the pointing between different maps), and the choice
of pixel size can also affect the beam, we have chosen not to try and
calibrate into units of the beam.


\section{Bowling}
Examples of the bowling in our maps.

\section{Noise}
\begin{itemize}
\item Analysis of the noise maps, how believable they are.
\item High noise towards bright sources.
\end{itemize}



\section{Catalogues}
\begin{itemize}
\item Fellwalker! \citep{2015A&C....10...22B}
\item Island and peak paradigm overview.
\item Details of our chosen parameters
\item Cross referencing between neighbouring tiles.
\end{itemize}

\subsection{Source recovery with makemap}
cf Steve Mairs paper. \citep{2014ApJ...783...60M} ?

\subsection{Comparison including fake sources?}

\subsection{Size Scales}

\section{Comparison with legacy surveys}

\section{Accessing the Legacy Release}
A brief summary of how to access the data, catalogues, any cool tools
we've found/developed before the release etc.

\section{Conclusions}

\section*{Acknowledgments}

The James Clerk Maxwell Telescope has historically been operated by
the Joint Astronomy Centre on behalf of the Science and Technology
Facilities Council of the United Kingdom, the National Research
Council of Canada and the Netherlands Organisation for Scientific
Research. This work was funded by the Science and Technology Facilities
Council.

\bibliography{legacy-850um-paper}
\bibliographystyle{mn2e}

\label{lastpage}
\bsp
\end{document}
