\documentclass[usenatbib]{mn2e}
\usepackage{mathpazo}

\usepackage{graphicx}
\usepackage{marginnote}
\usepackage{microtype}

\newcommand{\aspconf}{ASP Conf. Ser.}
% bold micron (for titles etc.
\newcommand{\bmicron}{\boldmath\ensuremath{\micron}}
% alternate micron command
\newcommand{\um}{\micron}


% commands and packages that shouldn't be needed in final version
\usepackage{xcolor, soul}
\newcommand{\todo}[1]{\textcolor{red}{TODO: #1}}
\newcommand{\note}[1]{\textcolor{red}{Note: #1}}

\title[SCUBA-2 850\um\ Legacy Release]%
{The JCMT Legacy Release: SCUBA-2 850\bmicron\ observations}

\author[S.~F.~Graves et~al.]
{Sarah~F.~Graves,$^{1,2}$
Graham~S.~Bell,$^{1,2}$
JAC
and
Others.\\
$^1$Joint Astronomy Centre, 660 N.\ A`oh\=ok\=u Place, Hilo, HI 96720, USA\\
$^2$East Asian Observatory, 660 N.\ A`oh\=ok\=u Place, Hilo, HI 96720, USA
}


\begin{document}

\date{in development}

\pagerange{\pageref{firstpage}--\pageref{lastpage}} \pubyear{2015}

\maketitle

\label{firstpage}

\begin{abstract}
  % Re write this properly once the rest of the paper is written.
  The JCMT Science archive allows access to raw and reduced data for
  all publicly available JCMT observations. To aid users of JCMT
  products, we have produced a standardised reduction of public JCMT
  SCUBA-2 850\micron SCUBA-2 observations. These reductions provide
  uniformly reduced coadds of of all data that was public as of August
  1st 2014, and catalogues to identify the regions where emission was
  detected. The data have been gridded in the HEALPix HPX scheme.
\end{abstract}

\section{Introduction}
\begin{itemize}
\item Description of the JSA \citep{2015Economou}
\item Very brief description of heterodyne and SCUBA-2 instruments, cite instrument papers. \citep{2013MNRAS.430.2513H} \citep{2009MNRAS.399.1026B}
\item Outline of the advanced data products. \citep{2014SPIE.9152E..2JB}
\item Historical perspective. \citep{2011ASPC..442..203E}
\item ongoing plans -- dynamic archive, continually updated etc.
\end{itemize}

% What data are included.
This release includes publicly-released 850\um\ observations observed
between 2011 February 2 and
2013 August 1. Observations from earlier than this were not included, as
these were taken in \emph{shared risk} mode while instrument
commissioning was still being conducted, and the data from that era
have more problems \citep{SC19}.  Observations by the JCMT Cosmology Legacy Survey
\citep{2013MNRAS.432...53G} are not included as they are still
proprietary. Pointing observations and observations from within the
appropriate time period that were judged to be of a low or
potentially-low quality \emph{were} reduced using this standardised
configuration\footnote{and are available for interested parties to
  download from the JSA}, but have not been included in the co-added
products.

\subsection{Overview of Release}
% Number of observations included.

% Area covered.
% Number of islands of emission and peaks.

% Picture of areas covered in this release (noise map?)





\section{HEALPix grid }
Require a fixed tiling and grid to reduce our data onto, as we want to
be able to incorporate all data current and future into this
method. The chosen grid is that of the HEALPix method (Hierarchical
Equal Area isoLatitude pixelation), commonly used in cosmological
fields. This has the advantage that all pixels have the same area. The
trade off is that the pixels do not have the same size in RA and Dec.

\todo{GSB: Insert Brief Description of healpix, advantages and
  disadvantages and why it was chosen }

\todo{Create plots illustrating healpix}.

\todo{list parameters we have used}.

\footnote{To re-grid a HEALPix tile onto a standard RA-Dec projection,
  the Starlink \textsc{smurf} command \texttt{jsajoin} can be used
  \citep{SUN258}.}

The Main HEALPix reference is \citep{2005ApJ...622..759G}

Also see: ``Mapping on the HEALPix grid'' is \citep{2007MNRAS.381..865C}



\subsection{SCUBA-2 reduction configuration}

\citep{2013MNRAS.430.2513H}

\begin{itemize}
\item Brief description of how SCUBA-2 data reduction works.
\item mapmaker paper: \citep{2013MNRAS.430.2545C}
\item Details of our chosen configuration and why we chose it. (detailed).
\item Examples of some reduced maps.
\end{itemize}


\subsection{QA}

\subsubsection{Standard JCMT QA States}
All observations taken by the JCMT are classified as 'GOOD' (the
default), 'BAD', 'QUESTIONABLE', 'JUNK' or 'REJECTED'. We included
GOOD, QUESTIONABLE and REJECTED data in our co-adds. The REJECTED
state is used by the JLS teams to indicate that a particular
observation did not meet their particular QA criterion but the data
are otherwise usable. The QUESTIONABLE
state was originally designed to be a transient state that would be
resolved into either BAD or GOOD after analysis, but in practical
terms there was not a work flow to ensure this, so some observations
have this flag in the archive.

Usually (in the standard nightly reduction pipelines) QUESTIONABLE
data is treated as if it is BAD for the purposes of co-adds. However,
due to our second QA stage for this release, we chose to include it as
long as it passed our special legacy QA

\subsubsection{Legacy Release QA}
%Description of process. Examples of observations we threw out.

Every reduced file included in this has been marked as GOOD by a
member of the JAC team. This quick 'by-eye' assessment was simply done
on a image of the reduced map. This QA process was designed to avoid:
a) `blooms' or `blobs' of fake emission that can sometimes be produced
at the edges of the map by the mapmaker.  b) identify the most
problematic observations that had missed being flagged under the
telescope standard QA process.

This sort of process is of course subjective, and some of the
observations that we excluded might turn out fine on further
examination. However, as with all of this process, the focus for this
release has been to pursue ensure a high quality, even at the expense
of completeness. All of the HEALPix-gridded legacy-reduced
observations are available to the community in the JSA, regardless of
our QA analysis of them.





\subsection{Co-adding}
Brief?
\begin{itemize}
\item Do we include anything to cope with hot pixels etc.
\end{itemize}



\section{Calibration}
\begin{itemize}
\item Calibration paper \citep{2013MNRAS.430.2534D}
\item show images the standard calibrators.
\item comparison with 'standard' calibrator reductions
\item effect of pixel size on fluxes.
\item effect of pointing errors on fluxes.
\item Our uncertainty in the final flux.
\end{itemize}


We calibrated our data into units of mJy per arc-second squared, using
the standard FCF values from \citet{2013MNRAS.430.2534D}.  As the
effective beam size is increased during the co-add procedure (due to
errors in the pointing between different maps), and the choice of
pixel size will also affect the effective beam size of the final map,
we have chosen not to try and calibrate into units of the beam.


\section{Bowling}
Examples of the bowling in our maps.

\section{Noise}
\begin{itemize}
\item Analysis of the noise maps, how believable they are.
\item High noise towards bright sources.
\end{itemize}



\section{Catalogues}
\begin{itemize}
\item Fellwalker! \citep{2015A&C....10...22B}
\item Island and peak paradigm overview.
\item Details of our chosen parameters
\item Cross referencing between neighbouring tiles.
\end{itemize}


\section{Robustness of products}
% NB improve this section title.

\subsection{Source Recovery: Comparison including fake sources?}
cf Steve Mairs GBS paper (in prep)
\note{ probably merge this and previous subsection}

\subsection{Size Scales}

\section{Comparison with legacy surveys}

\section{Accessing the Legacy Release}
A brief summary of how to access the data, catalogues, any cool tools
we've found/developed before the release etc.

\section{Conclusions}

\section*{Acknowledgments}

The James Clerk Maxwell Telescope has historically been operated by
the Joint Astronomy Centre on behalf of the Science and Technology
Facilities Council of the United Kingdom, the National Research
Council of Canada and the Netherlands Organisation for Scientific
Research. This work was funded by the Science and Technology Facilities
Council.

\bibliography{legacy-850um-paper}
\bibliographystyle{mn2e}

\appendix
\onecolumn

%% Probably more detailed than needed in a paper
\section{Mapmaker configuration}
The mapmaker configuration used for this release (known as a
\texttt{dimmconfig}) is shown here. Like most SCUBA-2 mapmaker
configuration files, it first sources the \emph{base}
\texttt{dimmconfig} file that sets up the basic values for a range of
options and then tweaks a subset of additional parameters or its
purposes. This file is shipped in the 2015A Starlink release. The
value of every configuration parameter used is written into the
history component of the file.

\begin{verbatim}
^$STARLINK_DIR/share/smurf/dimmconfig.lis

#  Less aggressive cleaning to cope with bright sources
noisecliphigh=10.0
dcthresh = 100

#  Don't want extended structure, so avoid problems with COM model by using
#  individual common-mode models for each subarray.
com.perarray = 1

#  Aggressive filtering.
flt.filt_edge_largescale=200

#  Allow bolometer noise to vary with time, and using a box filter to
#  determine mean noise in each box, in order to presevre as many samples
#  as possible.
noi.box_size=-15
noi.box_type=1

#  Use a maximum of 20 iterations
numiter=-25
maptol_mean=1
maptol=0.01

# new recommendations and using an ast model
ast.zero_snr = 5
ast.zero_snrlo = 3

ast.skip=5
flt.zero_snr=5
flt.zero_snrlo=3
\end{verbatim}


\label{lastpage}
\bsp


\end{document}
