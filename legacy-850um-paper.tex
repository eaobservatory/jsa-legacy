\documentclass[times]{aastex61}


\usepackage{underscore}
% alternate micron command
\newcommand{\um}{\micron}

% referencing commands
\newcommand{\sref}[1]{Sec.~\ref{#1}}

\usepackage[T1]{fontenc}

% commands and packages that shouldn't be needed in final version
\usepackage{xcolor, soul}


\newcommand{\ascl}[1]{\href{http://www.ascl.net/#1}{ascl:#1}}
\newcommand{\status}[1]{\textsf{#1}}
\newcommand{\jyas}{Jy\,arcsec\textsuperscript{$-$2}}
\newcommand{\jybm}{Jy\,beam\textsuperscript{$-$1}}

%\hypersetup{bookmarks=true}

\shorttitle{JCMT LR: SCUBA-2 850\micron}
\shortauthors{SF Graves et al.}


\begin{document}
\title{The JCMT Legacy Release: SCUBA-2 850\micron\ Coadds and Catalogs}

\author[0000-0001-9361-5781]{Sarah F. Graves}
\affiliation{East Asian Observatory, 660 N.\ A`oh\=ok\=u Place, Hilo, HI 96720, USA}
\affiliation{Joint Astronomy Centre, 660 N.\ A`oh\=ok\=u Place, Hilo, HI 96720, USA}
\author[0000-0003-0438-8228]{Graham S. Bell}
\affiliation{East Asian Observatory, 660 N.\ A`oh\=ok\=u Place, Hilo, HI 96720, USA}
\affiliation{Joint Astronomy Centre, 660 N.\ A`oh\=ok\=u Place, Hilo, HI 96720, USA}
\author{David S. Berry}
\affiliation{East Asian Observatory, 660 N.\ A`oh\=ok\=u Place, Hilo, HI 96720, USA}
\affiliation{Joint Astronomy Centre, 660 N.\ A`oh\=ok\=u Place, Hilo, HI 96720, USA}
\author[0000-0002-7316-4626]{Iain M. Coulson}
\affiliation{East Asian Observatory, 660 N.\ A`oh\=ok\=u Place, Hilo, HI 96720, USA}
\affiliation{Joint Astronomy Centre, 660 N.\ A`oh\=ok\=u Place, Hilo, HI 96720, USA}
\author{Malcolm J. Currie}
\affiliation{RAL Space, Rutherford Appleton Laboratory, Harwell Campus, Didcot OX11 0QX, UK}
\affiliation{Joint Astronomy Centre, 660 N.\ A`oh\=ok\=u Place, Hilo, HI 96720, USA}
\author{Jessica T. Dempsey}
\affiliation{East Asian Observatory, 660 N.\ A`oh\=ok\=u Place, Hilo, HI 96720, USA}
\affiliation{Joint Astronomy Centre, 660 N.\ A`oh\=ok\=u Place, Hilo, HI 96720, USA}
\author{Per Friberg}
\affiliation{East Asian Observatory, 660 N.\ A`oh\=ok\=u Place, Hilo, HI 96720, USA}
\affiliation{Joint Astronomy Centre, 660 N.\ A`oh\=ok\=u Place, Hilo, HI 96720, USA}
\author[0000-0001-5982-167X]{Tim Jenness}
\affiliation{Joint Astronomy Centre, 660 N.\ A`oh\=ok\=u Place, Hilo, HI 96720, USA}
\affiliation{Large Synoptic Survey Telescope, 950 N. Cherry Ave, Tucson, 85719, USA}
\author{Doug Johnstone}
\affiliation{NRC Herzberg Institute of Astrophysics, 5071 West Saanich Rd, Victoria, BC, V9E 2E7, Canada}
\affiliation{Department of Physics and Astronomy, University of Victoria, Victoria, BC, V8P 1A1, Canada}
\affiliation{Joint Astronomy Centre, 660 N.\ A`oh\=ok\=u Place, Hilo, HI 96720, USA}
\author{Harriet A. L. Parsons}
\affiliation{East Asian Observatory, 660 N.\ A`oh\=ok\=u Place, Hilo, HI 96720, USA}
\affiliation{Joint Astronomy Centre, 660 N.\ A`oh\=ok\=u Place, Hilo, HI 96720, USA}
\author{Mark G. Rawlings}
\affiliation{East Asian Observatory, 660 N.\ A`oh\=ok\=u Place, Hilo, HI 96720, USA}
\author{Holly S. Thomas}
\affiliation{Joint Astronomy Centre, 660 N.\ A`oh\=ok\=u Place, Hilo, HI 96720, USA}
\affiliation{Harvard-Smithsonian Center for Astrophysics, 60 Garden Street, Cambridge, MA 02138, USA}
\author[0000-0002-4694-6905]{Jan G. A. Wouterloot}
\affiliation{East Asian Observatory, 660 N.\ A`oh\=ok\=u Place, Hilo, HI 96720, USA}
\affiliation{Joint Astronomy Centre, 660 N.\ A`oh\=ok\=u Place, Hilo, HI 96720, USA}

\correspondingauthor{Sarah F. Graves}
\email{s.graves@eaobservatory.org}

\begin{abstract}
  We present the JCMT 850\,\micron{} Legacy Release, containing
  uniformly reduced coadded tiles, and catalogs of detected emission, for the
  850\,\um\ data from all SCUBA-2 observations taken between 2011
  February 2 and 2015 March 1.  This release provides the fastest and
  easiest way to identify uniformly determined 850\,\micron\
  detections and calibrated fluxes for any position observed by the
  JCMT. The coadded observations cover 1356 square degrees of the sky,
  with detections of contiguous areas of emission at better than
  5-$\sigma$ covering 1.4 square degrees of area. 13477 individual
  local maxima were detected within these regions. Sixteen tiles
  contain regions with a noise better than 1\,m\jybm.

  The data are gridded onto HEALPix tiles of $\approx$1 degree a side,
  using the HPX projection with pixels of size $\approx$3.22
  arcseconds.  The coadds have been calibrated into units of m\jyas\
  using a single, self-derived flux conversion factor (FCF) of 2.46
  Jy\,pW$^{-1}$arcsec$^{-2}$. We discuss in detail the derivation of
  our FCF and the limitations of our calibration.

  The coadds and catalogues can be searched at CADC at:
  \url{http://www.cadc-ccda.hia-iha.nrc-cnrc.gc.ca/en/search/?Observation.proposal.id=JCMT-LR}.
  Combined catalogs of all detected regions and of all local maxima
  can be downloaded from \url{doi:INSERT\ URL\ HERE}, along with masks
  showing the full area observed in this release and the full area of
  detected emission.
\end{abstract}

\keywords{astronomical databases:miscellaneous, catalogs; submillimeter:general}


%Silliness to get two column figure at start...)
\section{Introduction}
\label{sec:intro}
\onecolumngrid
\begin{figure*}[h!]
  \centering
  \includegraphics{mollweide-average-noise-galacticaxes-crop}
  \caption{All-sky noise map for this SCUBA-2 850\,\um{} legacy
    release. Each pixel represents the average noise across a single
    HEALPix tile. Tiles with fewer than 200 valid pixels were not
    included. The map is a Cartesian Mollweide projection, with the
    axes of the galactic coordinate system overlaid. The large curved
    stripe shows the extensive mapping of the Galactic Plane. Small
    points away from the plane predominantly represent small scanning
    (DAISY) observations towards extragalactic sources scattered
    across the sky.}
  \label{fig:noise-mollweide}
\end{figure*}
\twocolumngrid


% JCMT & SCUBA-2 & JSA
The James Clerk Maxwell Telescope (JCMT) is a 15-m submillimetre
telescope located at an altitude of 4092\,m on Maunakea, where it has
been operating since 1987. Its current instrument suite includes
SCUBA-2 (Submillimetre Common User Bolometer Array 2), a 10,000-pixel
continuum camera that observes simultaneously at 450\,\um\ and
850\,\um\ wavelengths \citep{Holland2013}, and has been operating in
full science mode since February 2011.

% why this release
To maximize the scientific return of these many years of archival
data, and to make it as easy as possible for non-submm-experts to use
the JCMT data, the Joint Astronomy Centre (JAC), then-operators of the
JCMT, decided to produce `legacy' reductions of all public data from
the current instrumentation. This work has been continued since 2015
March by East Asian Observatory (EAO), the current operators of the
JCMT. These legacy reductions are envisioned as providing a uniform,
standardized reduction, coadds, and emission detection of all publicly
available observations, regularly updated as more observations become
public. The aim has been to produce uniformly reduced high-quality
coadds maps that required as little checking as possible by eye, and
which would allow astronomers to easily discover which regions have
been observed, the noise levels in existing regions, and where there
are clear detections. Previously, the SCUBA Legacy Catalog
\citep{DiFrancesco2008} presented uniformly reduced coadds and
catalogs of much of the public SCUBA data, and this resource has been
very widely used.

The first product of the legacy release project, the JCMT legacy
release (LR1) was made public in September 2015, and included
850\micron\ SCUBA-2 data from 2011 February 2 to 2013 August 1. This
paper presents the expanded and updated second legacy release,
including SCUBA-2 850\micron\ data from 2011 February 2 to 2015 March
1st. This covers all observations taken before the handover of the
JCMT from JAC to EAO. For simplicity, we will refer to the current
release as the LR.  It is expected that the 850\,\micron\ legacy
release will be regularly updated with new versions in the future, as
more SCUBA-2 data is taken and becomes public.

Planned future releases are the 450\,\um\ SCUBA-2 observations and
HARP spectral cubes, as well as regular updates to the 850\,\um\
release.

% Added back in -- oculd be shorter, but description and citation of
% JSA needs to go somwhere.
This data release is publicly accessible and fully queryable by sky
position through the JCMT Science Archive \citep[JSA]{2015Economou},
hosted at the Canadian Astronomy Data Centre (CADC). This archive also
hosts our proprietary (with authentication) and public raw
observations and our standard pipeline reduced products. Unlike the
LR, the normal reductions do not include full coadds of data taken on
different nights. In addition, these observations from different
projects may have been reduced with different configurations and can
not be naively coadded.


\subsection{Outline of paper}

This paper presents first in section \ref{sec:overview} a summary of
the release and the included observations, various overall statistics
and some details on the noise properties of the release. Section
\ref{sec:examples} shows the scientific products created towards three
example regions (chosen to cover a range of different astronomical
objects). Section \ref{sec:healpix} describes the HEALPix grid used in
this release, and then Section \ref{sec:dr} describes our data
reduction procedures. We describe the calibration of these maps in
Section \ref{sec:calib}, including a discussion of some of the more
subtle effects of the data reduction method used. Section
\ref{sec:cat} then describes the emission catalogs. We finish by
summarizing the pointing offsets found in Section \ref{sec:pointing},
and the beam shape in Section \ref{sec:beam}.

\section{Overview of Release}
\label{sec:overview}
This release includes publicly available 850\,\um\ science
observations observed between 2011 February 2 and 2015 March 1,
including observations from calibration projects, PI projects, and the
JCMT Legacy Surveys (JLS). See \citealt{ChrysostomouJLS} for an
overview of the JCMT Legacy Survey program, which included SCUBA-2
observations for: the Gould Belt Survey \citep{GBS}, the SCUBA-2
ambitious Sky Survey \citep{SASSy}, the Cosmology Legacy Survey
\citep{Geach2013}, the JCMT Galactic Plane Survey \citep{JPS}, the
Nearby Galaxy Survey \citep{NGLS}, and the SONS Debris Disk Survey
\citep{Holland2017}.  Observations from earlier than 2011 February 2 were not
included, as these were taken in \emph{shared risk} mode while
instrument commissioning was still being conducted, and the data from
that era are more problematic \citep{SC19,Dempsey2010}. The first
block of the released data, from 2011 to 2013 August 1 (but not
including the Cosmology Legacy Survey (CLS) observations) was
previously reduced and publicly released in September 2015, using the
same configuration and coadd criteria as described in this paper, but
calibrated using a slightly different flux conversion factor (FCF).


For these observations, we have produced individual reduced
observations, coadds of all science observations that fell on a given
tile, and catalogs of extended contiguous regions and local peaks
detected at $>5\sigma$. Fig.\,\ref{fig:noise-mollweide} shows the
all-sky noise map of the coadds in this release, showing the average
RMS across all valid pixels in a tile as a single square.

All pointing and science observations from our time range, (except
Solar System objects/moving targets, which are not included in this
release), were reduced using the ORAC-DR \citep{2015oracdr} recipe
\texttt{REDUCE\_SCAN\_JSA\_PUBLIC}\footnote{ORAC-DR is the data
  reduction pipeline used at the JCMT} onto our tiles, using the HPX
projection (see Section \ref{sec:healpix}). These reduced maps are in
the instrumental units of pW (see Section \ref{sec:dr}). The
individual-observation maps are available to download through the JSA,
where they are listed under their original project code and
metadata. Pointing observations were not used or analysed further in
this release.



All the reduced science observations were individually examined to
determine if they met our quality standards. Coadds were then made for
every tile with data falling onto it, using the recipe
\texttt{COADD\_JSA\_TILES} (see Section \ref{sec:coadd}), producing a
coadded tile calibrated in units of m\jyas{} (for details of the
calibration see Section~\ref{sec:calib}). For every coadded tile, the
recipe \texttt{JSA\_CATALOGUE} was then run to produce (if detected)
extent and peak catalogs of detected emission (see Section
\ref{sec:cat}).

The observations included in this release were each taken in one of
the two different SCUBA-2 standard scan patterns: CV\_DAISY (hereafter
DAISY), a fixed-size scan pattern used for covering small areas, and
CURVY\_PONG (hereafter PONG) scan patterns at various user-chosen
sizes, used for covering larger areas \citep{Bintley2014}.  The
scanning speed is not the same between these scan patterns; this will
cause some inconsistency in the scales of emission detected in the
reduced maps, as SCUBA-2 observations are sensitive to different size
scales of emission at different speeds. However, our reduction has
been extremely conservative and has filtered out structure on scales
larger than 200\arcsec, so this difference will be minimized in our
coadds.



\subsection{Anticipated uses}
The primary aims of this release were: to allow users to see where we
have observed; to identify the detected emission; and to get
consistently calibrated fluxes from the detected emission. In
particular, we wanted to make these task extremely \emph{easy} to carry out,
especially for astronomers who are not experts in submillimeter
observations.

Simplifying common tasks, such as extracting fluxes and identifying
detections for a large catalog of objects, is necessary to enable many
large-scale or multi-wavelength analyses, where there is no scope for
the researchers to reduce data from every project, instrument and telescope
themselves. Easy access to detected calibrated fluxes and maps is also
invaluable as pathfinders for follow-up observations (for example,
identifying sources for high resolution follow up with ALMA).



% An easy way to identify what has been observed by the JCMT and find
% its noise level (by searching for coadded tiles by position and
% downloading them).

% Easy access to find robust detections of 850um emission by the JCMT for an arbitrary list of observations:
% i.e. if you have a list of sources and need to identify ones with
% 850um detections.

% Easy extraction of calibrated fluxes towards a list of positions for
% SED modelling.

% pathfinder/finder chart for follow up high res observations with e.g. ALMA.

% Multiwavelength analysis inputs: easy access to 850um calibrated
% detections/limtis of detections for large source lists, especially
% for users who are not experts in sub-millimeter observations.

% Pathfinders for future work; particulary when preparing/planning
% future observations or large scale analysis, easy access to
\subsection{Accessing the release}

The data products forming this JCMT Legacy Release are fully included
in the CADC's multi-observatory archive (which includes the JSA), and
will return as appropriate in searches based on spatial, temporal and
spectral limits. This is hosted at
\url{http://www.cadc-ccda.hia-iha.nrc-cnrc.gc.ca/en/search}.

Our uncalibrated reductions of single observations can be found by all
the same criteria which would identify the raw and normal reductions
of that observation -- including telescope, proposal code, PI name,
target name, as well as the spatial, temporal and spectral
constraints.

In addition, users can search specifically within the coadded tiles
and the catalogs by searching for proposal ID \texttt{JCMT-LR} along
with any other desired constraints, or to see the all coadds and
catalogs at once they can go directly to
\url{http://www.cadc-ccda.hia-iha.nrc-cnrc.gc.ca/en/search/?Observation.proposal.id=JCMT-LR&Observation.Collection=JCMT}
%Please note that you will find there are extent-850um results shown
%for all tiles, even those where no emission was detected.

All of these searches can also be done either through the web
interface or through the VO-standard TAP protocol \citep[which enables
more complex queries]{2010ivoa.spec.0327D}. The CADC TAP service can
be easily queried from within TOPCAT \citep{2005ASPC..347...29T}.

Appendix \ref{sec:filetypes} summarises the scientific content
included in each file type produced for this release, along with the
productID under which they are listed within the JSA, the naming
scheme for the file and the software that created the file.


For convenience, we have also provided a single-file download
combining all of the extent and peak catalogs on the EAO/JCMT website
at: \url{INSERT\ URL\ HERE}. That page also contains spatial masks
outlining the full area observed in this release, as well as the full
area of the detected emission.


\subsection{Overall statistics of the release}

In total, 21464 observations were reduced representing 6420 hours of
observing time, towards 4403 distinct tiles. The coadded observations
include 12404 science and calibration observations which passed QA
(5828 hours or 242.8 days) towards 4151 tiles. These coadds were made
up of 136 hours of science calibration observations, 3575 hours of
data taken for the JCMT Legacy Surveys and 2117 hours of data taken
for PI projects. This information is summarized in Table
\ref{tab:typesobs}. The coadds include 1.69 gigapixels of valid data,
corresponding to $\approx$ 1356 square degrees. For comparison, the
SCUBA Legacy Catalog \citep{DiFrancesco2008} covered 19.6 square
degrees in their `Fundamental Map Data Set, and 29.3 square degrees in
their fuller `Extended Map Data Set'.

\begin{deluxetable}{lrrr|rrr}

  \tablecaption{Types of observation included in this release.\label{tab:typesobs}}
  \tablecolumns{7}
  \tablehead{
    & \multicolumn{3}{c}{Indiv. Obs} & \multicolumn{3}{c}{Coadds}\\
    \cline{2-4} \cline{5-7}
    \colhead{Type} & \multicolumn{1}{p{0.75cm}}{Num. obs.} & \multicolumn{1}{p{0.75cm}}{Time (h)} & \colhead{Tiles} & \multicolumn{1}{p{0.75cm}}{Num. obs.} & \multicolumn{1}{p{0.75cm}}{Time (h)} & \colhead{Tiles}}
  \startdata
All & 21464 & 6420.0 & 4403 & 12404 & 5828.1 & 4151 \\
Sci. & 12594 & 5914.5 & 4185 & 12404 & 5828.1 & 4151 \\
Pointing & 8870 & 505.5 & 292 & 0 & 0.0 & 0 \\
Calib. & 2234 & 138.1 & 22 & 2200 & 135.9 & 22 \\
JLS & 6143 & 3642.2 & 2156 & 6023 & 3575.3 & 2145 \\
PI & 4217 & 2134.2 & 2329 & 4181 & 2117.0 & 2303\\
\enddata
\end{deluxetable}

Emission detection was carried out on all of the coadded tiles, and
this resulted in detections in 1.37 square degrees
($4.17\times10^{-4}$ steradians), or 0.10\% of the area observed).


\begin{deluxetable}{rDDD}
  \tablecaption{Areas of the release with an RMS noise less than or
    equal to the given value. \label{tab:noises}}
  \tablehead{\colhead{RMS} &
    \multicolumn{2}{p{1.6cm}}{Equivalent} &
    \multicolumn{2}{p{1.25cm}}{\centering Area} &
    \multicolumn{2}{p{1.25cm}}{\centering Fraction}\\
    \colhead{\centering $\mathrm{mJy\,beam^{-1}}$} &
    \multicolumn{2}{p{1.5cm}}{\centering $\mathrm{mJy\,arcsec^{-2}}$}&
    \multicolumn{2}{p{1.25cm}}{\centering $\mathrm{{}deg^{2}}$} &
    \multicolumn{2}{p{1.25cm}}{$ $}}

  \decimals
  \startdata
  1.0 & 0.0043 & 0.1819 & 0.00013 \\
  2.0 & 0.0086 & 0.8075 & 0.0006 \\
  10.0 & 0.043 & 44.76 & 0.033 \\
  25.0 & 0.107 & 176.7 & 0.13 \\
  50.0 & 0.215 & 331.6 & 0.24 \\
  100.0 & 0.431 & 1043 & 0.77 \\
  200.0 & 0.862 & 1277 & 0.94 \\
  \enddata
\end{deluxetable}

\begin{figure}
  \centering
  \includegraphics{coadds-noise-histogram.pdf}
  \caption{Histogram of the RMS noise on each pixel in the coadds. The
    main plot is shown with a log $y$-axis scale, and a linear
    $y$-axis scale is shown in the inset. Note that no trimming of
    observations is done for this, so the noisy edges of the SCUBA-2
    observing patterns are included. The noise is taken from the
    \texttt{makemap} produced variance array, which uses the scatter
    of input data points to estimate the variance of a given pixel
    while reducing the raw time series onto a sky map.}
  \label{fig:histonoise}
\end{figure}

Because the observations used here are from a wide variety of projects
aiming for a variety of noise levels, our noise maps are extremely
heterogeneous. Figure~\ref{fig:noise-mollweide} shows the all-sky
distribution of the noise maps of our
coadds. Figure~\ref{fig:histonoise} shows a histogram of the noise in
each pixel of the coadded tiles. It can be seen that the most common
noise is 0.35\,m\jyas\ (81\,m\jybm). Table~\ref{tab:noises} shows the
area of our coadds which has a noise level below various common
values. This is calculated by finding the number of pixels which have
a noise value less than or equal to a given limit. We have given the
noise in \jyas\ and Jy\,beam$^{-1}$, calibrated using our derived FCF
of 2.46\,mJy\,arcsec$^{-2}$pW$^{-2}$ (see Section \ref{sec:calib} for
details).
\subsection{The deepest maps}
\begin{deluxetable*}{r c r c c c c r}
  \tabletypesize{\footnotesize}
   \tablecaption{The sixteen coadded tiles that contain regions with
     a noise equal to or less than 1\,m\jybm. \label{tab:deepmaps}}
   \tablehead{%
     Tile &
     Area &
     Pixels &
    Time&
    \multicolumn{2}{c}{Mean RMS}&
    Sources &
    Number.\\
    \cline{5-6}
    &(deg$^{2}$) &
    &
    (h\,pix$^{-1}$) &
    (m\jyas)&
    (m\jybm)&
    &
    Obs.
    }
\startdata
 1238 & 0.0011 & 1369 & 5.94 & 4.04E-03 & 9.38E-01 & CRL\,618 & 923 \\
 1244 & 0.0010 & 1239 & 5.95 & 4.05E-03 & 9.39E-01 & CRL\,618 & 923 \\
 6383 & 0.0136 & 16942 & 4.58 & 3.72E-03 & 8.63E-01 & MACS0717.5+3745 & 89 \\
8921 & 0.0006 & 770 & 3.77 & 4.22E-03 & 9.80E-01 & MACSJ1423.8+2404 & 44 \\
11252 & 0.0306 & 38233 & 6.94 & 3.23E-03 & 7.50E-01 & GOODS-N-CANDELS, CDF-N & 221 \\
11425 & 0.0304 & 38024 & 6.35 & 3.12E-03 & 7.25E-01 & AEGIS-CANDELS& 260 \\
12640 & 0.0004 & 484 & 3.67 & 4.24E-03 & 9.83E-01 & MACSJ2153.6+1741, Abell 2390& 85 \\
12642 & 0.0087 & 10914 & 4.76 & 3.74E-03 & 8.68E-01 & MACSJ2153.6+1741, Abell 2390& 85 \\
13297 & 0.0030 & 3720 & 5.56 & 3.95E-03 & 9.16E-01 & CRL\,2688 & 911 \\
17679 & 0.0010 & 1252 & 3.57 & 3.90E-03 & 9.05E-01 & UKIDSS-UDS-CANDELS& 871 \\
17690 & 0.0268 & 33503 & 5.56 & 3.33E-03 & 7.74E-01 & UKIDSS-UDS-CANDELS& 871 \\
17741 & 0.0029 & 3665 & 4.83 & 3.95E-03 & 9.16E-01 & Abell\,370& 73 \\
26189 & 0.0036 & 4497 & 3.45 & 3.89E-03 & 9.04E-01 & Abell\,1689& 45 \\
27258 & 0.0458 & 57261 & 9.84 & 2.63E-03 & 6.11E-01 & COSMOS-CANDELS& 948 \\
28448 & 0.0024 & 2970 & 3.64 & 3.99E-03 & 9.27E-01 & MACSJ1149.5+2223& 43 \\
35935 & 0.0100 & 12495 & 5.86 & 3.79E-03 & 8.80E-01 & CDF-S& 115\\
\enddata
\tablecomments{This table lists the tile numbers and the
  representative source names. It also lists some statistics
  calculated on the set of pixels in each tile which have an RMS
  $\leq$1\jybm: the area, the number of pixels, the average exposure
  time per pixel, and the mean noise across these pixels. The deepest
  tile in this list, 27258, is also shown in Figure~\ref{fig:t27258}.}
\end{deluxetable*}



Twenty of our coadded tiles contain pixels with a noise better
0.0043\,m\jyas{}, corresponding to 1\,m\jybm. These include our three
most common standard SCUBA-2 calibrators (CRL\,618, CRL\,2688 and
Arp\,220), and several deep cosmology fields that were observed
separately by both the JCMT Cosmology Legacy Survey and by various PI
projects. For reference, these twenty tiles, along with some
statistics of their deep regions are listed in
Table\,\ref{tab:deepmaps}.



\section{Example regions}
\label{sec:examples}
This data release includes emission towards objects covering the full
range of non-solar-system structures observed by the JCMT --
including dusty disks, complex filamentary molecular clouds, nearby
galaxies, extragalactic point sources, and extremely deep surveys of
standard cosmological fields. Although providing full examples of all
types of emission is beyond the scope of this paper, we here present
some of our coadded tiles covering a range of source types.

\subsection{G034.27+0.15}
The source G034.27+0.15 is a bright, high-mass star-forming region
that has been observed in a wide variety of wavelengths, including
\citet{Hajigholi2016,Avalos2009,Mookerjea2007,Rigby2016}. Figure~\ref{fig:g34-3}
shows our coadded tile, noise map, extent catalog, and peak catalog
towards this source. This coadd includes data primarily from JCMT
calibration projects of this source, but also two observations from a
University of Hawai`i (UH) PI project (M13AH07B). The noise maps
(Figure~\ref{fig:g34-3}, 2nd figure) for this coadd shows the
characteristic DAISY scan pattern towards the central source, with two
overlapping small PONG observations covering a wider extent. This
field also illustrates the higher noise we see towards extremely
bright sources -- clearly seen in the towards the central source and
the bright points in the filament extending towards the top
left. Examination of the extent and peak catalogs, however, (shown
below) indicates that this not prevent a detection of the emission in
these regions. Our experience has been that as higher-noise areas only
occur towards extremely bright regions, they are still clearly
detected in our extent catalogs, as they have a signal-to-noise ratio
(SNR) of $>5$.



% includes a variety of types of structure three examples shown here
% to examine the co-adds and the catalogs.  Includes: filamentary
% region (G34.3), bright point source (CRL~618, a standard calibrator)
% and the deepest map in our data set, a long observation of what is a
% 'blank field in any single observation, but reveals a multitude of
% deep submm galaxies when co-added.


\begin{figure*}[htb!]
  \centering
  \includegraphics{tile30318-g34-coadd-noise.pdf}
  \\[3mm]
  \includegraphics{tile30318-g34-extent-peak.pdf}
  \caption{The coadded products towards tile\,30318, containing the
    high-mass star-forming source G034.27+0.15. Upper left: the
    coadded emission map. Upper right: the coadded noise map,
    exhibiting the characteristic noise patterns of both the smaller
    DAISY observations in the centre, and the larger PONG observations
    around them. Lower left: the detected extents of emission,
    outlined on the coadded emission map. Lower right: the detected
    peaks within the detected extents. The extent and peak catalogues
    are described fully in Section \ref{sec:cat}. }
  \label{fig:g34-3}
\end{figure*}

\subsection{CRL\,618}
CRL\,618 is an extremely bright and well-studied proto-planetary
nebula, and is one of the JCMT's standard secondary flux calibration
sources for SCUBA-2. For a recent overview of it, see
e.g. \citet{Soria-Ruiz2013}. It is observed frequently with short
DAISY observations for calibration monitoring during normal nightly
science observing. As such, this coadd includes 923 observations using
51 hours of observing time across the small DAISY area, shown in
Figure~\ref{fig:crl618}. We achieved an extremely low noise due to
this large number of repeats -- achieving a noise better than
0.005\,m\jyas\ around the central source. A magnified view of the area
around the central position shows clearly a variety of point sources
detected in this deep coadd. In addition, it can readily be seen by
eye that there appear to be additional sources that are missed by our
cataloging procedure due to the negative bowling around the very
bright source. Horizontal and vertical cut throughs are shown to help
visualise this bowling.

\begin{figure*}[htb!]
  \centering
  \includegraphics{crl618-whole-map.pdf}
  \\[3mm]
  \includegraphics{crl618-sourceonly.pdf}
  \caption{Examples of the coadd, noise map, and extents towards the
    standard calibration source CRL\,618, consisting of tiles\,1238
    and 1244. This coadd contains data from 923 observations (51\,hrs
    elapsed observing time). Top left: the intensity map. Top right:
    the noise map, where the contours indicate the 0.005, 0.006, and
    0.007 m\jyas\ noise levels. Bottom left, a magnified area around
    the central source is shown with the detected extents around the
    source as white contour. Bottom right: the horizontal and vertical
    cuts through the central peak, at the positions indicated by the
    dashed lines on the bottom left plot. The upper red line has been
    offset from 0 by +0.02\,m\jyas\, and the lower cyan line by
    -0.02\,m\jyas, to allow both to be clearly seen on one plot. In
    these cuts the $\approx$ 0.02 m\jyas\ bowling around the bright
    central source can be easily seen. }
  \label{fig:crl618}
\end{figure*}

\subsection{Tile\,27258: Deep COSMOS-CANDELS field}
Here we present our deepest map. This includes data both from the JCMT
Cosmology Legacy Survey \citep[recently published by][]{Geach2016} but
also from various University of Hawai`i deep surveys
\citep{Casey2013,Chen2013,Chen2013a}. By being able to include
publicly available data taken by different projects we can achieve a
deeper map than possible with only one data set. Our coadd includes
both wide PONG observations, and very deep DAISY
observations. Figure~\ref{fig:t27258} shows both the entire tile, its
noise map, and also a close in view of the deepest region. In the noise map
it is possible to see an imprint of the original observations scan
patterns (both PONG and DAISY observations are included in this
tile); no attempt to correct for this was made. In this deep field
it can also be seen that there is no artifical structure appearing at
the edges of the different observations; this indicates that our
variance arrays (used to weight the input data at the coadding stage)
are accurate even in these high-noise areas.


In this deep map, it can clearly be seen that there are point sources
spread across the full map. Although for optimal detection of point
sources a beam matched filter should be used, even without using that
we detect a large number of these objects. Figure~\ref{fig:t27258}
shows a comparison of our detections, the CLS's detections from their
first data release \citep{Geach2016} and those of
\citet{Casey2013}. Although these three methods do not detect exactly the
same objects, there are many correspondences between them.

\begin{figure*}[htb!]
  \includegraphics{27258-whole-map.pdf}
  \\[3mm]
  \includegraphics{27258-zoomin.pdf}
  \caption{Tile\,27258: an extremely deep coadd towards a
    COSMOS field. This tile comprises both PONG and DAISY observations
    taken by the JCMT Cosmology Legacy Survey (CLS) and several UH cosmology
    projects. Parts of these data were published in
    \citet{Casey2013,Chen2013,Chen2013a,Geach2016}. Top left: emission
    map of the entire coadded tile. A white contour indicates the
    extremely deep area with a noise < 5\,\jyas\, occurring because of
    many DAISY observations of this area.  Top right: noise map of
    the entire tile. Contours indicate the 1.75, 2, 2.5, 5, 7.5, and
    10 m\jyas\ levels. Bottom left: an enlargement of the deepest
    region of the map, showing the 850\micron\ data. The white contour
    indicates the same < 5m\jyas\ noise region. Bottom right:
    comparison of the source catalogs produced here (shown as red
    crosses), as published in \citealt{Casey2013} (black circles) and
    as found by the JCMT CLS 850um DR1 \citep[purple
    triangles]{Geach2016}, across the same enlarged region.}
  \label{fig:t27258}
\end{figure*}


\section{HEALPix grid}
\label{sec:healpix}

To produce this uniform reduction of all public data, it was necessary
to have a suitable scheme for dividing the sky into pre-determined
tiles and pixels onto which to grid the data. The scheme needed to be
well-defined in advance, without reference to the position of existing
observations, for consistency and in order to be able to easily
incorporate further data as they become public.  The tiling scheme
chosen was HEALPix \citep[Hierarchical Equal Area isoLatitude
Pixelization,][]{Gorski2005}, commonly used by cosmologists. The
HEALPix system starts by dividing the sky into twelve facets, and then
recursively divides these cells in four at each higher resolution
level.  We use the \emph{nested} numbering scheme to
label our tiles.% this scheme allows for simple conversions to and from
%positions in the HPX projection, as well as compatibility with Virtual
%Observatory systems such as MOC \citep[Multi-Order
%Coverage,][]{2013ASPC..475..135F}.

Pixels within the individual tiles use the HPX projection
\citep{Calabretta2007}, which allows us to use a (higher resolution)
HEALPix grid within our standard 2-D arrays in a FITS files. This
ensure that the pixelization is continous between each tile and the
adjacent tiles, and hence neighbouring tiles can be joined simply by
abutting them.  The details of the HEALPix parameters used are listed
in Table~\ref{tab:hpxpar}.

While HEALPix has the advantage that all pixels have the same area,
the trade off is that the pixels are not generally square: they can
vary in width and length while maintaining the constant area.  There
are also discontinuities in the angle of the grid lines at some facet
boundaries.  This is largely a display problem, as although image
viewers with a modern WCS implementation can handle files in the HPX
projection, the sources will appear distorted (or non-circular) due
the non-square pixels, and grid lines will appear bent at
discontinuities.

\begin{table}
\centering
\caption{HEALPix parameters used in this release.\label{tab:hpxpar}}
\begin{tabular}{r l}
\tableline
Tile size & $\sim$ 1\degr \\
Tile $N_\mathrm{side}$ & 64 \\
Pixels per tile  & 1024 $\times$ 1024\\
Pixel size &  $\sim$ 3.22\arcsec\\
Pixel $N_\mathrm{side}$ & $2^{16}$\\
\tableline
\end{tabular}
\end{table}

\section{Data reduction}
\label{sec:dr}
SCUBA-2 observations are reduced using the Starlink SMURF program
\texttt{makemap} \citep{Chapin2013}, called by the ORAC-DR pipeline
\citep{2015oracdr}. In brief, \texttt{makemap} uses an iterative process
that divides the raw data into models of various instrumental and
astronomical signals.  For full details on the SCUBA-2 map maker and
its models, see \citet{Chapin2013}, or for practical information on
using the software and the effect of different configurations in
practice, see Starlink Cookbook 21 \citep{SC21}. A detailed comparison
between the configuration used here and the Gould Belt Survey
reduction (tuned to recover larger scale emission) was shown in
\citet{Mairs2015}.

There are an extremely large number of user-adjustable parameters that
can affect the final map, depending on the science goals of the users
and the structure of the astronomical emission observed.%  For example,
% a cosmologist interested in detecting only faint point sources in a
% primarily blank map would use a distinctly different set of parameters
% from a galactic astronomer looking at bright, extended and
% structurally complex emission.
% reword: previous existing configs weren't appropriate to be used on
% all types of emission. We developed one that was, with a prime
% considerationb being avoided 'blooms' or 'blobs' of fake emission
To guide users, the JCMT-supported Starlink software comes with a
series of standard configuration files for a variety of common science
cases.  Unfortunately, these standard configuration files for
\texttt{makemap} can produce very poor results when used on an
inappropriate observation -- for example, the standard SCUBA-2
configuration file used for calibrator observations is tuned to expect
a bright, compact source at the centre of the map. If used by mistake
on a blank field this recipe could easily create \emph{fake emission}
in the form of large bloom-like structures. Avoiding this type of
problem was the paramount consideration in developing the
configuration for this release. We found we could minimise the blooms
of fake emission by not attempting to recover the large scale
structures in the map ($>$200\arcsec{}). When testing this
configuration we discovered it produced significantly better maps than
some of the default values we were using at the time, and we now use
this configuration as our default option in ORAC-DR when projects do
not select a more specific option.


These observations were reduced using Starlink, Starjava, and ORAC-DR
software developed between versions 2014A and 2015A. Interested
parties seeking to replicate these results can do so by using
the version of the code tagged as `2015A-legacy' in the
Starlink code repositories.\footnote{
  \url{https://github.com/Starlink/starlink},
  \url{https://github.com/Starlink/starjava} and
  \url{https://github.com/Starlink/ORAC-DR}.}

The exact configuration file used in the legacy reduction is shown and
discussed in Appendix~\ref{app:config}. There are a few specific
points we would like to highlight. We explicitly filter out
any size scales larger than 200\arcsec. We also calculate a separate
common-mode model for each subarray, and therefore are not sensitive
to any size scales smaller than a subarray.  Interested readers are
directed to \citet{Mairs2015}, which shows some examples of the same
data reduced with the JCMT Gould Belt Survey's reduction (tuned to
recover large scale structure) and with the JCMT-LR configuration.

Our configuration for \texttt{makemap} constrains the model of the
astronomical signal (the AST model) via an auto-generated spatial AST
mask, which identifies all regions (at that iteration) with a
signal-to-noise ratio $>5$ (followed down to 3) as containing
astronomical signal. See Sections \ref{sec:sourcerecovery} and
\ref{sec:astmask-fcf} for more detailed discussion of this.

We also set a convergence limit of either 25 iterations, or a change
between iterations of only 1\% of the noise level, whichever occurs
first. Only 4\% of the science observations did not converge within
the 25 iterations; these were still included in the coadds.

%\subsection{Quality Assurance}
%label{sec:QA}
Every reduced observation suitable for inclusion in the coadds was
assessed by eye, and not included in the coadds if was considered low
quality.  All of the legacy-reduced observations are publicly
available through the JSA, regardless of our QA analysis of them, so
interested users can individual evaluate each observations towards a
tile if they are dissatisfied with the coadd in the release.


\subsection{Coadding of tiles}
\label{sec:coadd}
All non-pointing observations for a given tile that passed QA were
coadded and calibrated using the PICARD recipe
\texttt{COADD\_JSA\_TILES}. PICARD is part of ORAC-DR, and performs
pipeline tasks on reduced data products. This recipe uses the
\texttt{makemos} routine from Starlink's CCDPACK package
\citep[][\ascl{1403.021}]{SUN139}. This coadding was carried out as a
variance-weighted mean. No despiking was performed, and no clipping of
the noisy edges of observations was done; no evidence was seen that
these procedures were required. The variance map for the coadded tile
was also produced by \texttt{makemos}, based on the variance of each
input observation. The calibration from pW into m\jyas\ was done by
multiplying by a Flux Conversion Factor of
2.46$\times$10\textsuperscript{3}\,m\jyas\,pW\textsuperscript{$-$1},
derived from our own data set -- see Section \ref{sec:calib}.

\subsection{Source Recovery}
\label{sec:sourcerecovery}
Recently, work by \citet{Mairs2015} examined the recovery of sources
found inside and outside the \texttt{makemap} AST mask (see Section
\ref{sec:dr}), specifically comparing the JCMT Gould Belt Survey's
reductions with the Legacy Release configuration described here. They
found that in coadds of SCUBA-2 reduced observations there is a
significant difference between recovery of sources inside and outside
the AST mask. Sources that are outside the AST-masked regions can have
their flux considerably under-represented. The AST masked region in a
single observation can be thought of as approximately the region
containing detectable emission in that observation, using `detection'
limits set by the specific configuration parameters chosen. The effect
is dependent on the size of the source -- extended sources outside of
the masked regions are poorly recovered in this reduction, whereas
point sources are better recovered.  This issue does (in theory)
affect all \texttt{makemap} reductions, including those using external
masks defined from previous knowledge of the emission. However, it is
more problematic for reductions such as these, in which the AST mask
is defined based only on detected emission from a single observation.

In case users of this release are concerned about the recovery of
specific sources, each individual reduced observation contains a bit
mask, stored in a FITS extension labeled QUALITY. This indicates which
areas of the map were included in the AST and/or FLT masks.



\subsection{Negative Bowling}
Within the coadded maps of this release, negative bowling can be
seen around bright sources. Figure~\ref{fig:crl618} shows this
around the bright calibrator CRL\,618. The cuts in the vertical and
horizontal direction through the source show the bowling at
$\approx$0.02\,m\jyas{}. As the size scales of the negative bowling are
comparatively large compared with small compact/point sources, this
could in theory be corrected for on a per-field basis. For this
release, no processing was done to try and ameliorate these effects,
and there will therefore be some loss of detected emission.


\section{Calibration}
\label{sec:calib}
The calibration of these data follows the basic approach used in the
standard SCUBA-2 calibration paper \citep{Dempsey2013}. In brief, a
Flux Conversion Factor (FCF) is derived from observations of standard
point-source calibrators of known brightnesses, which is then used to
convert from the instrumental units of pW into astronomical flux
density units. Measuring the flux within a fixed-radius aperture
around the source allows conversion into mJy\,arcsec$^{-2}$ (an
aperture FCF), and a fit to the peak flux in the source allows
converting into mJy\,beam (a beam FCF). For this work, we derive both
an aperture and a beam FCF. However, we have calibrated all of our
maps (and all fluxes in our catalogs) only using the aperture FCF as
this is more appropriate for the extraction of flux in our catalogs of
detected emission.

% Underlying concept: for a given instrumental performance and dish
% shape, there is a single correct FCF to convert from a pW reduced
% map into the real source flux density. We measure this across a wide
% range of observations to identify the mean and range of values seen,
% and assume that our science observations instrumental performance
% and dish shape will follow the same range of variation as seen in
% our calibration observations.



Three commonly used SCUBA-2 calibrators were analysed when deriving an
FCF for this data set: Uranus, CRL\,618, and CRL\,2688.\footnote{CRL
  618 and CRL 2688 observations that had been marked as poor quality
  during QA were not included in this analysis.} The Uranus
calibration observations are not part of the standard legacy release
themselves, as observations of moving targets were not included in
this release. They were reduced onto the same area pixels using the
same DR configuration as our release.

For the aperture FCF, \citet{Dempsey2013} used a source aperture
radius of 30\arcsec{} and a background annulus of radii 45--60\arcsec,
which excluded approximately 8 percent of the Uranus flux. To select
our source aperture, we created curve-of-growth plots from deep
coadded maps of our calibration sources (Fig.~\ref{fig:cog}). We
selected a source radius of 60\arcsec\ (chosen by eye to avoid much
of the negative bowling seen around CRL 618 and CRL 2688, while not
missing too much of the flux from Uranus, which extends out further),
and a background annulus from 105-120\arcsec\ (chosen to avoid the
ringing structure detected in the deep coadds).

\begin{figure}
  \centering
  \includegraphics{cog-relative60arcs.pdf}
  \caption{The curve-of-growth for the deep coadded maps of Uranus,
    CRL\,618 and CRL\,2688. The top plot shows all flux included
    within a radius of the given size, normalised to the flux for that
    source in an aperture of 60" radius. The bottom plot shows the
    average flux per pixel in an annulus at the given radii,
    normalised to the flux in the first annulus. The 60\arcsec\ radius
    and the 105-120\arcsec background annulus used in the aperture
    FCF derivation are indicated on the plot.}
  \label{fig:cog}
\end{figure}



Using these apertures, the maps were masked with the appropriate
circles/annulus and the source and background statistics were
calculated using the ATOOLS and KAPPA Starlink packages. For the beam
FCF, the KAPPA command \texttt{beamfit} was used to fit a Gaussian
beam to the source and identify the amplitude.

The FLUXES package \citep{2014ascl.soft05010J} was used to derive the
expected flux of Uranus at the time of each observation. For CRL\,618
and CRL\,2688, the integrated and peak fluxes measured in
\citet{Dempsey2013} were used as the known values (see
Table~\ref{tab:fluxes}).

The FCF for each observation was then calculated using:
\begin{equation}
  \label{eq:1}
  \mathrm{FCF_{aperture}} = \frac{S_{\mathrm{integ}}}{I_{0}A}\ \mathrm{Jy\,pW^{-1}\,arcsec^{-2}}
\end{equation}

\begin{equation}
  \label{eq:2}
  \mathrm{FCF_{beam}} = \frac{S_{\mathrm{peak}}}{I_{\mathrm{peak}}}\ \mathrm{Jy\,pW^{-1}\,beam^{-1}}
\end{equation}

Where $S_{integ}$ is the known integrated flux in the source and
$S_{peak}$, $I_{0}$ is the summed flux in the aperture of the map
(minus the average flux per pixel in the background annulus multiplied
by the number of pixels in the aperture), $I_{peak}$ is the peak
amplitude of the \texttt{beamfit} Gaussian fit to the map, and $A$ is
the area of a pixel in the map ($3.22^2\,\textrm{arcsec}^{2}$).

The histograms of the derived FCFs are shown in Figure
\ref{fig:fcfhist}.% It can be seen that there is a much wider range of
%values found for CRL\,618; its not known why this is the case.

\begin{deluxetable}{r r@{$\pm$}l r@{$\pm$}l }
  \tablecaption{Fluxes used to calculate the FCFs. \label{tab:fluxes}}
  \tablecolumns{3}
  \tablehead{ \colhead{} & \twocolhead{$S_{\mathrm{integ}}$ (Jy)}
    & \twocolhead{$S_{\mathrm{peak}}$ (Jy\,beam$^{-1}$)}}
%    \colhead{}
%    & \twocolhead{(Jy)}
%    & \twocolhead{(Jy\,beam$^{-1}$)}}
  \startdata
  CRL\,618 & 5.0\phantom{0} & 0.2 & 4.89 & 0.24\\
  CRL\,2688& 6.13 & 0.21 & 5.64 & 0.27\\
  \enddata
  \tablecomments{ Fluxes taken from \citet{Dempsey2013}. The flux of
    Uranus (not shown), was found from the Starlink package FLUXES,
    based on the time of the observation.}
\end{deluxetable}


\begin{figure*}
  \centering
  \includegraphics{fcf-histogram.pdf}
  \caption{Histograms of the derived beam and aperture FCFs
    separately for CRL618, CRL2688 and Uranus. The combined histogram
    for CRL 618 and CRL2688 (i.e. the sources used to derive the
    calibration of the coadds) is shown as a grey background. The
    sigma-clipped mean for each source is shown as the vertical line,
    and the horizontal cross bar indicates the $\pm$ one standard
    deviation. The dashed line indicates the median value.}
  \label{fig:fcfhist}
\end{figure*}


\subsection{Effect of AST masking on derived aperture FCF}
\label{sec:astmask-fcf}
\begin{figure}
  \centering
  \includegraphics{toymodel-astmasking.pdf}
  \caption{A toy model illustrating the effect of source brightness on
    fraction of flux within and without an auto-generated AST mask
    (extending down to a limit of 3 times the noise). Two 1-D Gaussian
    sources are shown, at the (rough) intensities of Uranus (left) and
    CRL 618 (right). A noise level equivalent to a 10-$\sigma$
    detection of CRL 618 is shown in red. The shaded areas indicate
    the flux that would be found within the AST masked region, and the
    text indicates what percentage of the total flux in the source
    model this is.}
\label{fig:toymodel}
\end{figure}


% basic premise.
To convert from instrumental pW into flux densities (Jy),
\citet{Dempsey2013} used an average FCF derived from observations of
multiple bright calibrator sources.  They found the same FCF
distribution regardless of which calibrator source was
examined. However, our use of an auto-generated AST mask alters the
relative flux recovery in our different calibrator sources, causing
different FCF distributions to be found for different calibrator
sources. In this section, we look at how this affects our calibration
accuracy.


% As mentioned previously, \citet{Mairs2015} identified the differential
% recovery of sources inside and outside an AST mask when coadding
% multiple observations. Sources bright enough to be within the AST mask
% of a single observation have more of their 'true' flux recovered,
% compared with dimmer sources which are not contained within the AST
% mask. This affects the accuracy of the measured flux in science
% sources, but is not usually considered part of the calibration into a
% flux density.


%The recovery of source emission in SCUBA-2 observations reduced with
%\texttt{makemap} is strongly dependent on the exact details of the AST
%mask, unless the reduction configuration does not use one.
Our auto-generated AST mask follows sources down to a 3-SNR level;
therefore a point source will have the central part of its beam
included in the AST mask, and the outer areas not included. This
effect is illustrated in 1-D in Figure \ref{fig:toymodel}.  As shown
by \citet{Mairs2015}, the fluxes outside an AST mask are
suppressed compared to fluxes inside the AST mask; this means that
the emission in the outer regions of the beam will be suppressed
compared with the central region. As  Figure
\ref{fig:toymodel} shows, an extremely bright source (such as Uranus, with a
flux 57.2--70.3\,Jy for these observations) will have a higher
percentage of its emission found within the AST mask; a dimmer source
(such as CRL 618 at 5.0\,Jy or CRL 2688 at 6.13\,Jy) will have more of
its emission in the non-AST masked region where emission is
suppressed. This will lead to a systematic trend where, when comparing
two observations with the same noise, we will recover more of the flux
for brighter sources -- and hence derive a lower FCF. It can be seen
in Figure \ref{fig:fcfhist} that we do see a systematically lower FCF
distribution for our Uranus observations. We also believe this
variation in the AST mask causes the very different curves of growth
for Uranus as opposed to CRL\,618 and CRL\,2688 (see Figure
\ref{fig:cog}).

%\citet{Mairs2015} looked
%in some detail at a comparison between the recovery in flux of sources
%inside and outside the AST mask, for the LR \texttt{makemap}
%configuration. They found from simulations that the flux in sources
%outside the mask is suppressed compared with the 'true' value, and
%compared with sources inside the mask.


%In this reduction we use an
%auto-generated AST mask that finds sources with a 5-$\sigma$ detection
%at each iteration (and follows them down to a 3-$\sigma$ limit); this
%means that our AST mask, and hence flux recovery, is dependent on the
%brightness of the sources and the noise in the observation.



%However, we must also consider how the AST masking will change the the
%recovered flux in the \emph{calibration} source, as this alters our
%derived FCF. Generally, exploration of the AST mask to date has
%considered sources as entirely inside or entirely outside the AST
%mask, whereas in fact our auto-generated AST mask will cover the
%central (down to an SNR of 3) part of Gaussian point sources,
%and not the outer regions of the Gaussian.

%Fundamentally, the proportion of the flux found in the aperture around
%the source will be higher for a bright source, as the 3 SNR AST mask
%limit limit will include a larger percentage of the beam. In a dimmer
%source, a higher proportion of the flux will be found in the non-AST
%masked region, and will more of the flux will be suppressed in the
%final map. This is illustrated in 1-D in Figure \ref{fig:toymodel}.

%Therefore, for an observation reaching a similar noise level in the
%map, the auto-generated AST mask will have a different size for each
%source (based on their brightness -- see Figure
%\ref{fig:astmask}). This will then cause different proportions of the
%flux in the source to be recovered in the final map, due to the
%differing recovery in \texttt{makemap} reductions of flux within and
%without the beam.


% This has an important, but often ignored, effect on the flux recovery
% and derived FCFs from our calibration observations: very bright
% sources, such as Uranus, will have a larger area included in the AST
% mask due to a higher signal-to-noise at a given RMS, when compared
% with dimmer sources such as CRL\,618 and CRL\,2688.

 % Uranus has (from FLUXES) an
% integrated flux between 57.2 and 70.3 Jy for the observations used
% here, whereas CRL\,618 and CRL\,2688 have 5.0 and 6.13\,Jy
% respectively \citep[both from][]{Dempsey2013}.


%As a result, when performing aperture photometry at a specific source
%aperture, different proportions of the `true' total source flux will
%be found for sources of different brightnesses. This will cause a
%measurable difference in the measured FCF for the sources when using a
%constant aperture. In particular, Uranus observations have a much
%larger AST mask, and thus will have a higher proportion of their total
%integrated flux found at a specific aperture. This leads to a
%derivation of a smaller FCF for Uranus than for the dimmer calibration
%sources.

%For our analysis, this means that the derived FCF from the brightest
%point sources (i.e. Uranus) will be lower than that of the derived FCF
%from a dimmer point source (i.e. CRL 618 and CRL 2688), for an
%observation with the same noise level.



% bAsic Idea: we will calculate a different FCF from each source of a
% different brightness (or at different transmission), due to
% differing proportions of the source flux being found inside and
% outside the AST mask. This causes problems for the basic idea of
% calibrating a single FCF from a variety of sources of known flux --
% you won't get one value.






% The issue occurs due to the effect of the automatic AST masking used
% in the data reduction, combined with the extreme variation in the
% source brightnesses. In our configuration, an AST mask is produced
% each iteration around every source with an SNR peak greater than 5,
% and followed down to include all connecting pixels with an SNR greater
% than 3. If we compare the three calibration sources Uranus, CRL 618,
% and CRL 2688, these sources have significantly different
% brightnesses. CRL\,618 has an integrated flux of 5.0 Jy, CRL\,2688 is
% 6.13 Jy, whereas Uranus varies between 57.2 and 70.3 Jy for the
% observations in this analysis.





To test our understanding of the effect of source brightness, we
simulated 50 point sources at various peak brightnesses (in pW) from
0.002 pW to 0.1 pW, and reduced them using the \texttt{jsa\_generic}
configuration. This roughly ranges from a quarter of the peak
intensity of CRL\,618 (which has an average peak brightness of 0.009
pW in these observations) to roughly the brightness of Uranus (an
average peak of 0.11 pW). These maps were created using the
\texttt{fakemap} and \texttt{fakescale} options to \texttt{makemap},
with a source model taken from the central 200$\times$200 pixel square
found in a deep coadd of 800 Uranus observations, and using a single
blank DAISY observation for the SCUBA-2
data. Figure~\ref{fig:simulation} shows that we see a strong variation
in derived FCF (relative to the input simulated brightness) with the
input source brightness. We expect that this effect is primarily
governed by the relative levels of the source brightness
and the observation noise level, as well as the beam shape of the
input source.

Taken together, this means that we cannot expect to derive a single
FCF for sources of very different brightnesses, once we are trying to
achieve high calibration accuracies of better than
$\approx$10\,\%. Also, this will also affect our science data;
normally the suppressed flux outside the AST mask is considered as a
separate issue from the calibration uncertainty, but we have shown
here that these problems are linked. In an ideal world, one can
imagine applying different FCFs to areas inside and outside the AST
masks, based on detailed simulations and modeling. However, this would
be significantly beyond the scope of this work.

% These effects will, of course, also
%affect the science sources in this release. Normally we consider the
%concept of the flux suppression outside the AST mask as outside of the
%These concepts imply that for the data reduction method used here, we
%are reaching a limitation in our conception of a single FCF that
%calibrates a range of sources.
% Moves into concept of 'missing flux', not usually considered calibration.

%Pragmatic appraoch



% The work flow we followed from \citet{Dempsey2013} assumes that there
% is a single FCF value which will convert pW into a flux density for
% all emission, regardless of the source. However, because we have used
% an autogenerated AST mask in our data reduction, we have introduced an
% effect whereby our recovered flux in a calibrator is dependent on the
% brightness of the calibrator. Although slightly hidden by the natural
% variation across the observations, the aperture FCF histograms in
% Figure \ref{fig:fcfhist} do show that Uranus (which is far brighter)
% peaks at noticeably different



The AST masking issues would not be seen in the canonical SCUBA-2
calibration analysis presented in \citet{Dempsey2013}, as the special
\texttt{makemap} \texttt{bright\_compact} configuration used there
sets a fixed central AST aperture with a radius of 60\arcsec\ for all
observations, constraining all pixels beyond this radius to zero until
the last iteration.

% For reference,
%Figure~\ref{fig:astmask} shows the relationship between the area of
%the AST mask with the brightness of the simulated source.

% \begin{figure}
%   \centering
%   \includegraphics{astmask-peakbrightness.pdf}
%   \caption{The relationship between the area of the AST mask and the
%     brightness of the simulated source. The AST mask area is found
%     by counting the number of pixels included in the final AST mask,
%     and multiplying by the area of one pixel.}
%   \label{fig:astmask}
% \end{figure}


\begin{figure}
  \centering
  \includegraphics{simulated-fcfs.pdf}
  \caption{Simulated FCFs derived for 50 simulated point sources with
    peak brightnesses from 0.002--0.1\,pW, and using various aperture
    radii. The average peak brightness of our three calibration
    sources are indicated on the map. The input source was derived
    from a deep coadd of Uranus. The background map was a blank
    850\,\um\ DAISY observation. The simulated maps have an RMS noise
    of $\approx 2.7 \times 10^{-5}$ pW.}
  \label{fig:simulation}
\end{figure}

% plot of relative FCF for simulated bright source at various apertures and brightnesses

% plot of area included in AST mask at different brightnesses

% plot of relative FCF vs area in AST mask.
\subsubsection{Effect of transmission}
Due to the signal-to-noise limits in the AST masking, there will also
be a similar systematic effect in the measured FCF of a calibration
due to the RMS noise in the observation. Higher-noise observations
will have a smaller area in their SNR-defined AST mask, and thus a
higher FCF will be calculated. As the major determiner of the noise is
the 850\,\um\ transmission of the observation, this will lead to a
variation in derived FCF with both weather and elevation. Figure
\ref{fig:fcfairmass} shows the derived aperture FCFs plotted against
the 850\,\um\ transmission.

A spearman rank-order correlation test across all of our derived FCFs
found a correlation of $-0.44$ (pvalue=4$\times 10^{-122}$) between
the aperture FCF and the 850\,\micron\ transmission; for comparison,
the beam FCF (not expected to be strongly affected by the AST masking)
has a weaker correlation of $-0.32$ (pvalue=6$\times 10^{-59}$). Part
of this correlation is probably due to systematic errors in the
transmission measurement, which affect the extinction correction in
\texttt{makemap}. The difference between the beam and aperture FCF
correlations may be due to the AST masking effects.

For comparison, we also show the variation in derived aperture FCFs
with time. No correction to the applied FCF for sky opacity or date
was made in this release.

\begin{figure}
  \centering
  \includegraphics{fcf-extra.pdf}
  \caption{The variation of aperture FCFs with 850\,\um\ transmission
    (top) and with time (bottom). The aperture FCF used to calibrate
    this release, and its standard deviation, are shown at the solid
    black line and surrounding gray area. A clear (slight) trend can
    be seen between FCF and the transmission of the observation. Some
    extreme outliers have been cropped from the plots.}
  \label{fig:fcfairmass}
\end{figure}



%\hl{Please note that there is an additional complication here, in that the
%autogenerated AST mask actually varies in area \emph{at each
%  iteration}.}

\subsection{Derivation of mean FCF}

Despite the systematic issues with our FCF due to the AST masking
effects, we have chosen here for simplicity and consistency to
calibrate this data release using a single aperture FCF. As Uranus
is far brighter than anything in our released observations, we have
chosen to only use CRL\,618 and CRL\,2688 to derive our FCF value. We
used the derived FCF from each calibration observation described
above, then applied an iterative clipping which excludes FCF values
5-$\sigma$ away from the mean (as extreme outliers will often be due
to poor instrumental performance, bad weather, daytime observing or
other affects that are not present in the bulk of the released data.)
No explicit cut was made for the atmospheric transmission during the
calibrator observations, as no cut was made to the science
observations included in the coadds. Figure \ref{fig:fcfhist} shows
histograms of the full distributions for each source.  The final
average values found are:

\begin{eqnarray}
\mathrm{FCF}_{\mathrm{arcsec},850}&=&2.46 \pm 0.22\ \mathrm{Jy}\ \mathrm{pW}^{-1}\  \mathrm{arcsec}^{-2}.\\
\mathrm{FCF}_{\mathrm{beam},850}&=&571 \pm 48\ \mathrm{Jy}\ \mathrm{pW}^{-1}\ \mathrm{beam}^{-1}.
\end{eqnarray}

The coadds in this release were calibrated using this mean aperture
FCF. If a user of this release wishes to subsequently use a different
FCF calibration value, they can divide the coadded maps or catalog
fluxes by our standard FCF, and then recalibrate by multiplying with
their chosen FCF.


The standard deviation of the sample of derived FCFs represents the
minimum uncertainty in this calibration. In addition, the uncertainty
in the sub-mm flux of Uranus (used to bootstrap \citet{Dempsey2013}'s
measurements, and quoted as $\pm$ 5 percent therein), and the
uncertainty in the known fluxes of CRL\,618 and CRL\,2688 (see Table
\ref{tab:fluxes}. Taken together in quadrature, this gives an overall
estimated uncertainty in calibrated fluxes of $\pm$11 percent
(aperture) and $\pm$12 percent (beam) due only to the variation of
these derived FCFs.

For comparison, the usual `canonical' 850\,\micron\ aperture FCF used
in the JCMT's nightly reductions is $2.34 \pm 0.08 \mathrm{Jy}\
\mathrm{pW}^{-1}\ \mathrm{arcsec}^{-2}$ \citep{Dempsey2013}. This is
approximately 3 percent lower than the value used here, and was
derived using the `bright compact' \texttt{makemap} configuration on 1
arcsecond pixels (in comparison with the 3.22$\times$3.22 arcsecond
pixels in this release), using observations taken from 2011 May to
2012 May.

\subsection{Additional considerations in using the coadded fluxes}

\citet{Dempsey2013} derived appropriate FCFs for either point sources
or for circular aperture photometry, and we have followed their
approach here. However, it is clear that for the arbitrary shaped
regions of emission found in the catalogs of this release, it would
not be appropriate to perform circular aperture photometry. As
discussed above, this would also not be sufficient to correct for the
different proportions of flux included inside the AST mask.

As such, we present the integrated and peak flux found within our
sources, calibrated using the aperture FCF derived for CRL 618 and CRL
2688 using aperture photometry, without attempting to correct for
these effects or subtract any backgrounds. In practice, it should be
assumed that there is an additional significant (and not completely
known) uncertainty on fluxes presented here, or indeed on any flux
found by simple integration of flux found within an arbitrary
shape. From our analysis of simulations of point sources above, it is
likely that there are systematic trends where lower flux is more
poorly recovered than brighter flux. When highly accurate fluxes are
required, it may be necessary to perform detailed simulations
investigating the total flux recovery for a given AST mask, or to
re-reduce the data with a different reduction configuration
appropriate for emission structure.  We would also recommend users of
this release interested in this to investigate the AST mask found for
the individual observations that went into each mosaiced tile.


\section{Catalogs}
\label{sec:cat}
The JCMT Legacy Release includes emission catalogs generated from each
of the coadded maps. As we have an extremely diverse set of
astronomical regions (including blank fields, point sources, and
large, complex extended structures), we did not feel that it would be
possible to perform specific astronomical source finding and
classification within the scope of this release. Instead, we first of
all identified the \emph{regions of contiguous emission}, discussed
here as \emph{extents}, and secondly identified \emph{local maxima
  within those regions}, described here as \emph{peaks}. The primary
goal of these catalogs is to provide astronomers with regions of
securely detected emission. Images of both catalogs can be seen in
Figure \ref{fig:g34-3}.


We have chosen to use the FellWalker algorithm \citep{Berry2015}, as
implemented in the Starlink \texttt{cupid} \citep{cupid} package for
this analysis. Unlike the more commonly used ClumpFind, which simply
lays discrete contour levels on the map and uses those to identify
regions, FellWalker follows lines of ascent within the map to identify
all peaks of emission. We have found it to be robust and
easy-to-understand.

\subsection{Extents}
\label{sec:extents}
We have identified the regions of contiguous extended emission within
each tile by looking at the signal-to-noise maps of the coadded
tiles. We chose to consider all regions of contiguous emission
containing a pixel brighter than 5-$\sigma$, followed down to a limit
of 3-$\sigma$, to be a single `extent' for the purposes of our catalogs,
if the region is larger in area than a beam.  Here, a very simple
approximation of the beam was used: all sources comprising more than
nine pixels were kept. More complex approaches were considered,
but noise spikes did not appear to be commonly falsely detected as
emission with this simpler approach. %The main source of false detections in
%our reductions \emph{would} have been the previously discussed problem of
%`false blooms', if we had not used a by-eye approach for flagging the
%handful of problematic reductions and removing them from the coadds.

%The reduced observations, produced by SMURF's \texttt{makemap} routine, all
%contain a variance array giving a noise estimate for each pixel, based on
%the scatter of input data values to that pixel. Pixels with too few
%inputs to calculate a variance point are not included in
%the individual maps. Our coadding procedure uses this variance to
%weight the input observations.

We used the FellWalker algorithm as implemented in \texttt{cupid} on a
signal-to-noise (SNR) map to produce contiguous regions of emission
and to identify the position of the peak pixel within each map, and
these outlines are available both as a FITS format mask file, as a
HEALPix Multi-Order Coverage file \citep{2013ASPC..475..135F}. The
catalog for each tile indicates the ID of the clump, the RA and Dec of
the peak pixel's centre, the total flux contained within the entire
extent, the flux of the peak pixel, and the area of the entire
extent. The catalog also includes the approximation of the outline of
the extent as an STC-S \citep[Space-Time Coordinate
String]{2009ivoa.rept.1030R} polygon. An excerpt from the extent
catalog for tile\,30318 (containing G34.3) is shown in
Table~\ref{tab:extents}; the extents found in that tile are shown in
Figure \ref{fig:g34-3}.



\subsection{Peaks}
\label{sec:peaks}
In order to provide some information about the nature of emission
within the extents, this release includes identification of local
maxima within the extents. \emph{It is important to note that these
  should not in general be considered as point sources.} We identified
these maxima by running the FellWalker algorithm (from \texttt{CUPID})
on the regions we had previously identified as being in
\emph{extents}. The peak detections were not created from the SNR map,
as we could assume by only looking within the detected extents that we
were already looking at detected emission.

As our maps contain a varying noise, we adopted the mean noise across
the entire detected extent (from the variance array produced by
\texttt{makemos}) as a representative noise. We then ran the
FellWalker algorithm, using that representative noise as the RMS
value, and identified as individual local maxima all peaks with a
value of at least 5-$\sigma$, followed down to a noise level of
3-$\sigma$, and with at least a 5-$\sigma$ dip between them and their
neighboring peaks. Our output catalog contains the ID of the peak, the
RA and Dec of the peak pixel, the flux in the peak pixel, and the ID
of the extent the peak was found within. An excerpt from the peak
catalog for tile\,30318 is shown in Table~\ref{tab:peaks}; a plot of
this catalog is shown in Figure \ref{fig:g34-3}.

We chose not to include the outlines of all emission counted as
being within that object or clump, as we felt that would encourage
users to consider the peaks as being discrete physical objects. Although
they may be in some cases, in many other cases the peak is simply a
local maximum within a complex molecular cloud. The `clump outlines'
produced by algorithms such as FellWalker always reflect the arbitrarily
chosen boundaries of dips and minimum heights, and should not be
assumed to represent physical objects without considerable further
modeling.


\section{Pointing}
\label{sec:pointing}
\begin{figure}[htb!]
  \centering
  \includegraphics{pointing-offset.pdf}
  \caption{Histogram of the total magnitude of the position offsets
    found in the calibration observations (including pointings). The
    median value of 2.88\arcsec{} is indicated on the plot.}
  \label{fig:pointing}
\end{figure}

Our observations of JCMT standard calibrators and pointing sources are
corrected for normal pointing offsets, as these sources are all at
known positions. In our data reduction stage, these observations are
run through \texttt{makemap} twice; first to calculate the difference
between the beam position and the known source position, and secondly
re-reduced using this positional offset to place the observation at
the correct point. No other observations in this data set were
corrected for their positions.

The JCMT usually quotes a pointing offset of approximately 2\arcsec{}
in $x$ and $y$, corresponding to an average radial offset of
2.9\arcsec{}. By examining the offsets in RA and Dec found for the
calibrator and pointing observations, we identified that the average
pointing offset for the release matches this expected value, with a
median radial offset of 2.88\arcsec{}. See Figure~\ref{fig:pointing}
for the histogram of these offsets.  In general, we can assume that
the pointing in our non-calibration observations should be on average
\emph{more} accurate than this, as the JCMT operators will correct for
offsets found in a pointing observation before they continue with the
science observations. When examining coadded science observations, if
the beam area is required to a high accuracy than a correction factor
for beam smearing may need to be included.


\section{Beamshape}
\label{sec:beam}
\begin{figure*}[htb!]
  \centering
  \includegraphics{beamfit-uranus.pdf}
  \caption{The SCUBA-2 850\micron\ beam, as found from a deep coadd
    of Uranus observations. On the left is shown an image of the
    central source. The bright ring at ~160" radius is due to
    misaligned panels on the telescope. On the right: profiles in RA
    and Dec of the beam are shown with the fitted two-Gaussian beam
    model overlaid for contrast. The $y$-axis shows the value relative
    to the full amplitude of the beam.}
\label{fig:beam}
\end{figure*}

Using the same deep coadded maps of Uranus used in Section
\ref{sec:calib}, we fitted a two-component beam model to the JCMT
beamshape, and found the same shape and size as found in
\citet{Dempsey2013}. We use the KAPPA command \texttt{beamfit} to do
this, fitting two circular Gaussians with a fixed background level of
0. We found FWHM of 12.6\arcsec\ and 44.0\arcsec\, with a relative
amplitude of 96 percent and 4 percent. Figure \ref{fig:beam} shows the
image of the deep Uranus coadded maps, as well as a comparison
between the radial profiles along the horizontal and vertical axes and
the fitted beam shape.



\section{Summary}

We have presented here the JCMT legacy release of the SCUBA-2
850\,\micron\ observations taken from 2011 to 2015, including coadded
tiles and catalogs of detected emission. This release provides an easy
means of identifying robust detections across all regions observed by
the SCUBA-2, of finding calibrated 850\,\micron\ fluxes, and for
producing finder charts and fluxes for follow up high resolution
observations.

The coadded tiles and catalogs of this release are fully searchable in
the JCMT Science Archive via CADC's multi-observatory search page at
\url{http://www.cadc-ccda.hia-iha.nrc-cnrc.gc.ca/en/search/?Observation.proposal.id=JCMT-LR&Observation.Collection=JCMT},
or via their TAP service (e.g. from within TOPCAT). In addition, we
have provided single-file download of the combined catalogs, as well
as single-file masks (in MOC format) of the full area included in the
release, and of all detected emission in the release. These can be
found at \url{INSERT\_URL\_HERE}.

The coadded tiles include 5828 hours of observing time over 4151
HEALPix tiles, covering 1356 square degrees of the sky. Robust
detections of emission was found in 1.37 square degrees of that
area. Sixteen of the tiles contain extremely deep regions with a noise
better than 1\,m\jybm. A data reduction configuration was developed
for this release, which can be successfully used with both SCUBA-2
scanning modes and over a wide range of astronomical objects. We
examined every observation included in the coadds by eye, and removed
the small number that did not appear to be of sufficient quality.

We followed the method of \citet{Dempsey2013} and derived a single
Flux-Calibration-Factor (FCF) to convert all of our coadded tiles from
instrumental pW to m\jyas\, based on observations of the bright
sources CRL\,618 and CRL\,2688 and their known fluxes. We then
discussed in detail some of the limitations of this approach, caused
by the details of the data reduction configuration we used. This
configuration was required to allow disparate observations of
e.g. blank fields or bright filamentary clouds to be reduced with the
same setup.

From our calibrated, coadded tiles, we then used FellWalker on the SNR
map to robustly identify regions of emission for an 'extents'
catalog. Within those extents, we then ran FellWalker again on the
flux density map to extract local maxima and create our 'peak'
catalog.

We analysed a deep map of all the Uranus observations taken
during the time range of this release to (and reduced with the same
configuration as our release) to determine the beam shape of these
observations. We also examined the positions of our known calibrator
and pointing sources to determine the pointing offsets we expect to
see in our coadded tiles.


\acknowledgments
The James Clerk Maxwell Telescope has historically been operated by
the Joint Astronomy Centre on behalf of the Science and Technology
Facilities Council of the United Kingdom, the National Research
Council of Canada and the Netherlands Organisation for Scientific
Research. This work was funded by the Science and Technology
Facilities Council.  Additional funds for the construction of SCUBA-2
were provided by the Canada Foundation for Innovation.

The work presented here was begun by the Joint Astronomy
Centre, and latterly has been supported by East Asian Observatory (EAO).
EAO has operated the JCMT since 2015 March 1, on behalf of the
National Astronomical Observatory of Japan, Academia Sinica Institute
of Astronomy and Astrophysics, the Korea Astronomy and Space Science
Institute, the National Astronomical Observatories of China and the
Chinese Academy of Sciences (Grant No. XDB09000000), with additional
funding support from the Science and Technology Facilities Council of
the United Kingdom and participating universities in the United
Kingdom and Canada.

This research used the facilities of the Canadian Astronomy Data
Centre operated by the National Research Council of Canada with the
support of the Canadian Space Agency. Many thanks are due to the staff
of CADC for their help with various technical aspects of this release.

The authors wish to recognize and acknowledge the very significant
cultural role and reverence that the summit of Maunakea has always had
within the indigenous Hawaiian community.  We are most fortunate to
have the opportunity to conduct observations from this mountain.

 \vspace{5mm}
\facility{JCMT(SCUBA-2)}

% Comma separated list, include software citations here as \citep{}
% after name of software, before next comma.
\software{Starlink \citep{2014ASPC..485..391C,2011ascl.soft10012V},
  SMURF \citep{2013ascl.soft10007J},
  SMURF-makemap \citep{Chapin2013},
  KAPPA \citep{2014ascl.soft03022C},
  ORAC-DR \citep{2015oracdr,2013ascl.soft10001J},
  ATOOLS,
  CCDPACK \citep{2014ascl.soft03021W},
  CUPID \citep{2013ascl.soft11007B},
  FLUXES \citep{2014ascl.soft05010J},
  Astropy \citep{2013A&A...558A..33A}, % units, fits, tables, wcsaxes
  Healpy \url{https://github.com/healpy},
  PyMOC \url{https://github.com/grahambell/pymoc}
}

\bibliography{legacy-850um-paper}
\bibstyle{aasjournal.bst}




\clearpage
\appendix


\section{Mapmaker configuration}
\label{app:config}
The mapmaker configuration used for this release (known as a
\texttt{dimmconfig} file) is shown here. Like most SCUBA-2 mapmaker
configuration files, it first includes the \emph{base}
\texttt{dimmconfig} file that sets up the basic values for a range of
options and then tweaks a subset of parameters for its
purposes. This file is included in the 2015A Starlink release. The
value of every configuration parameter used is written into the
history component of the reduced file.

\begin{verbatim}
^$STARLINK_DIR/share/smurf/dimmconfig.lis

#  Less aggressive cleaning to cope with bright sources
noisecliphigh=10.0
dcthresh = 100

#  Don't want extended structure, so avoid problems with COM model by using
#  individual common-mode models for each subarray.
com.perarray = 1

#  Aggressive filtering.
flt.filt_edge_largescale=200

#  Allow bolometer noise to vary with time, and using a box filter to
#  determine mean noise in each box, in order to presevre as many samples
#  as possible.
noi.box_size=-15
noi.box_type=1

#  Use a maximum of 20 iterations
numiter=-25
maptol_mean=1
maptol=0.01

# new recommendations and using an ast model
ast.zero_snr = 5
ast.zero_snrlo = 3

ast.skip=5
flt.zero_snr=5
flt.zero_snrlo=3
\end{verbatim}



\subsection{Details of the legacy DR configuration}
In this particular reduction, after the initial pre-processing stage,
the SCUBA-2 mapmaker recipe performs five iterations in which the
astronomical model (AST) is retained at the end rather than being
removed as in a normal iteration. (This is specified using
\texttt{ast.skip=5}). These initial iterations allow the reduction
process to focus on creating a mask for the FLT model (which is used
to filter the low frequency noise) by identifying any unusually bright
region; these regions may cause ringing in subsequent estimates of the
low-frequency noise in each bolometer. During these iterations
``bright'' pixels are taken to be those with a signal-to-noise ratio
greater than 5.0, plus any other pixels that are attached contiguously
to such pixels down to an SNR of 3 (specified by using
\texttt{flt.zero\_snr=5} and \texttt{flt.zero\_snrlo=3}).

During each iteration, an atmospheric opacity correction is applied
and the data are filtered to remove any features on scales larger than
200\arcsec. In addition to this, a separate common-mode model is
produced for each subarray (specified with
\texttt{com.perarray=1}). Having a separate common-mode estimate per
subarray prevents artificial flux (also known as ``fake blooms'')
being introduced into the map where the common modes vary
significantly and are not well represented by a single common
mode. The disadvantage with having a separate common mode for each
subarray is that the reduction is limited to recovering emission
structures that are on the same scale or less than the subarray size
(approximately 200").

After the initial five iterations, the recipe then performs up to 20
further iterations with all the models (specified by
\texttt{numiter=-25}; the initial five iterations plus these
additional 20 iterations containing all models). The number of total
iterations was limited to 25 so that time to reduce all observations
in this release was not excessive. Each of these remaining iterations
removes the AST model of the astronomical signal from the residuals
prior to starting the next iteration. However, to prevent
instabilities in the iterative process, this subtraction only occurs
within regions corresponding to bright sources.  The AST mask
identifying such sources is created in the same way as the mask used
by the FLT model: source pixels are taken to be those with a
signal-to-noise ratio greater than 5.0 (plus any other pixels that are
attached contiguously to such pixels down to an SNR of 3 (specified by
using \texttt{ast.zero\_snr=5} and \texttt{ast.zero\_snrlo=3}).  The
map maker exits before the 20th iteration if the \texttt{maptol}
parameter -- the change between maps -- has reached a mean value of
1\% of the noise level (specified by \texttt{maptol=0.01}) Only 562 of
the science observations (4.4 percent) did not converge within the 25
iterations. These were not flagged and were included in the coadded
tiles.

\newpage
\section{Full listing of file types in this release}
\label{sec:filetypes}
\movetabledown=2in
\begin{rotatetable}
\begin{deluxetable}{llp{4.5cm}p{3.5cm}p{3.5cm}l}
%\tabletypesize{\scriptsize}
\tablewidth{\linewidth}

\tablecolumns{7}
\tablecaption{Listing of all types files contained in this data release.}
\tablehead{\colhead{Object} & \colhead{recipe} & \colhead{filename} & \colhead{Contained in file} & \colhead{Notes} & \colhead{productID}}
\startdata
Single observation
& \multicolumn{1}{p{2.5cm}}{REDUCE\_SCAN\_JSA\_PUBLIC}
& \texttt{jcmts<YYYYMMDD>\_<SCAN>\_850\_healpix<TILE>_obs_000.fits}&
\raggedright{\textbullet{} Emission map (pW)}\linebreak
\raggedright{\textbullet{} Variance map (pW\textsuperscript{2})}\linebreak
\raggedright{\textbullet{} Quality map (bit mask)}
& Multiple files per observation, one for each tile the observation fell onto & healpix-850um\\
%
Coadded tile &  COADD\_JSA\_TILES  & \texttt{jcmts850um\_healpix<TILE>\_pub\_002.fits} &
\raggedright{\textbullet{} Emission map (m\jyas)}\linebreak
\raggedright{\textbullet{} Variance map (m\jyas)\textsuperscript{2}}
 & One file per tile & healpix-850um\\
%
Extent catalog & JSA\_CATALOGUE&\texttt{jcmts850um\_extent-cat<TILE>\_pub\_002.fits}
&\textbullet{} Catalog for extents (see Sect.\,\ref{sec:extents})
& Only created if  emission detected at $>5\sigma$. & extent-850um\\
%
Extent mask & JSA\_CATALOGUE & \texttt{jcmts850um_extent-mask<TILE>\_pub\_002.fits} &
\textbullet{} Mask map indicating which pixels fell into which extent
& " & "\\
%
Extent MOC & JSA\_CATALOGUE & \texttt{jcmts850um\_extent-moc<TILE>\_pub\_002.fits} &
\textbullet{} MOC file, indicating which pixels fell into any extent & " & "\\
Peak catalog & JSA\_CATALOGUE & \texttt{jcmts850um\_peak-cat<TILE>\_pub\_002.fits} &
\textbullet{} Catalog of peaks (see Sec.\,\ref{sec:peaks})&
Only created if extents catalog created. & peak-850um\\
%
Coverage MOC & JSA\_CATALOGUE & \texttt{jcmts850um\_tile-moc<TILE>\_pub\_002.fits} &
\textbullet{} MOC file, identifying all pixels that contain valid data (as opposed to blank pixels) & Created for all tiles. & extent-850um\\
\enddata
\tablecomments{In the filename column, \texttt{<SCAN>} indicates the
  JCMT scan number of that observation, padded with zeros to five-digit
  length. \texttt{<YYYYMMDD>} indicates the UT date of the
  observation. \texttt{<TILE>} indicates the HEALPix tile number of
  that tile, using the nested scheme and the HEALPix parameters given
  in Table\,\ref{tab:hpxpar} ($N_\mathrm{side}$ = 64).
  \\
  The \emph{single observation} files will be found in the JSA under the
  original observation's metadata, observation ID, and project code. All
  remaining files are in the JSA in a tile-centric fashion, and are found under the project
  code `JCMT-LR', with metadata appropriate for the specific tile, and
  with an observation ID of the form \texttt{SCUBA-2-<TILE>}.}
\end{deluxetable}
\end{rotatetable}
\newpage
\section{Example Catalogs}
Excerpts from the extent and peak catalogs of tile\,30318.
\movetabledown=2in
\begin{rotatetable}
\begin{deluxetable}{c c c c c c p{7cm}}
\tablecaption{An excerpt from the catalog of extents for Tile 30318.\label{tab:extents}}
\tablehead{\colhead{ID} & \colhead{RA} & \colhead{DEC} & \colhead{TOTAL_FLUX} & \colhead{PEAK_FLUX} & \colhead{AREA} & \colhead{SHAPE}\\ \colhead{ } & \colhead{deg} & \colhead{deg} & \colhead{$\mathrm{mJy}$} & \colhead{mJy\,arcsec$^{-2}$} & \colhead{arcsec$^{-2}$} & \colhead{ }}
\startdata
JCMTPX_J185318.8+011459 & 283.3285 & 1.2497 & 5.415E+09 & 2.105E+02 & 1.633E+05 & Polygon ICRS TOPOCENTER 283.3003 1.256714 283.2408 1.289357 283.3113 1.273799 283.3003 1.308009 283.3347 1.306843 283.3283 1.386135 283.3019 1.392549 283.3381 1.488935 283.3406 1.306847 283.3964 1.189091 283.3532 1.215324 283.3594 1.166936 283.3271 1.152366 283.339 1.196665 283.2928 1.201333 \\
JCMTPX_J185334.7+011421 & 283.3944 & 1.2392 & 2.599E+06 & 1.647E+00 & 8.609E+03 & Polygon ICRS TOPOCENTER 283.3889 1.235147 283.3855 1.240392 283.3759 1.241558 283.3841 1.256706 283.3878 1.257872 283.391 1.268366 283.3944 1.268947 283.3958 1.260787 283.4051 1.254378 283.3996 1.246216 283.4026 1.238634 283.4102 1.231055 283.4109 1.222318 283.4074 1.221737 283.4061 1.226401 \\
JCMTPX_J185330.0+010240 & 283.3752 & 1.0445 & 5.236E+06 & 5.745E+00 & 7.148E+03 & Polygon ICRS TOPOCENTER 283.3608 1.032857 283.3603 1.036351 283.3624 1.039264 283.3612 1.04218 283.3621 1.049172 283.369 1.051884 283.3841 1.050334 283.3866 1.046261 283.3845 1.039849 283.3871 1.035768 283.3848 1.034608 283.382 1.029943 283.3738 1.023732 283.3669 1.029947 283.3635 1.030529 \\
JCMTPX_J185322.6+010831 & 283.3443 & 1.1419 & 2.774E+05 & 9.040E-01 & 2.702E+03 & Polygon ICRS TOPOCENTER 283.3363 1.137201 283.3392 1.141282 283.3408 1.148856 283.3491 1.15235 283.3518 1.150022 283.3523 1.147694 283.347 1.13429 283.3436 1.131768 283.3367 1.135266 \\
JCMTPX_J185321.5+010610 & 283.3395 & 1.1028 & 7.271E+05 & 1.185E+00 & 4.310E+03 & Polygon ICRS TOPOCENTER 283.3337 1.100475 283.3335 1.105721 283.3351 1.109802 283.3321 1.117381 283.3367 1.122421 283.3388 1.121457 283.339 1.117381 283.3461 1.102807 283.3468 1.098726 283.3447 1.094645 283.3472 1.088237 283.3381 1.083192 283.3347 1.085322 283.3347 1.088812 283.3392 1.09348
\enddata
\end{deluxetable}

\end{rotatetable}
\begin{deluxetable}{ccccc}
\tablecaption{An excerpt from the catalog of peaks for Tile 30318.\label{tab:peaks}}
\tablehead{\colhead{ID} & \colhead{RA} & \colhead{DEC} & \colhead{PEAK_FLUX} & \colhead{PARENT_EXTENT}\\ \colhead{ } & \colhead{deg} & \colhead{deg} & \colhead{mJy\,arcsec$^{-2}$} & \colhead{ }}
\startdata
JCMTPP_J185318.8+011459 & 283.3285 & 1.2497 & 2.105E+02 & JCMTPX_J185318.8+011459 \\
JCMTPP_J185318.2+012524 & 283.3257 & 1.4234 & 4.608E+01 & JCMTPX_J185318.8+011459 \\
JCMTPP_J185316.0+011518 & 283.3168 & 1.2550 & 3.018E+01 & JCMTPX_J185318.8+011459 \\
JCMTPP_J185318.7+012445 & 283.3278 & 1.4124 & 2.470E+01 & JCMTPX_J185318.8+011459 \\
JCMTPP_J185316.5+011434 & 283.3189 & 1.2427 & 2.004E+01 & JCMTPX_J185318.8+011459 \\
JCMTPP_J185321.6+011341 & 283.3401 & 1.2281 & 1.224E+01 & JCMTPX_J185318.8+011459 \\
JCMTPP_J185320.8+012825 & 283.3367 & 1.4736 & 7.523E+00 & JCMTPX_J185318.8+011459 \\
JCMTPP_J185319.8+011257 & 283.3326 & 1.2159 & 5.712E+00 & JCMTPX_J185318.8+011459 \\
JCMTPP_J185315.2+011713 & 283.3134 & 1.2870 & 4.566E+00 & JCMTPX_J185318.8+011459 \\
JCMTPP_J185318.2+011339 & 283.3257 & 1.2276 & 2.932E+00 & JCMTPX_J185318.8+011459 \\
JCMTPP_J185316.9+011318 & 283.3202 & 1.2217 & 2.896E+00 & JCMTPX_J185318.8+011459 \\
JCMTPP_J185328.2+011350 & 283.3676 & 1.2305 & 2.699E+00 & JCMTPX_J185318.8+011459 \\
JCMTPP_J185322.5+011211 & 283.3436 & 1.2031 & 2.474E+00 & JCMTPX_J185318.8+011459 \\
JCMTPP_J185326.9+011251 & 283.3621 & 1.2142 & 2.456E+00 & JCMTPX_J185318.8+011459 \\
JCMTPP_J185323.4+011137 & 283.3477 & 1.1937 & 2.224E+00 & JCMTPX_J185318.8+011459 \\
JCMTPP_J185324.3+011119 & 283.3511 & 1.1885 & 2.164E+00 & JCMTPX_J185318.8+011459 \\
JCMTPP_J185320.5+012224 & 283.3353 & 1.3733 & 2.119E+00 & JCMTPX_J185318.8+011459 \\
JCMTPP_J185307.1+011600 & 283.2797 & 1.2666 & 2.098E+00 & JCMTPX_J185318.8+011459 \\
JCMTPP_J185318.2+012722 & 283.3257 & 1.4561 & 2.053E+00 & JCMTPX_J185318.8+011459 \\
JCMTPP_J185326.9+011551 & 283.3621 & 1.2643 & 1.935E+00 & JCMTPX_J185318.8+011459 \\
JCMTPP_J185325.4+011318 & 283.3559 & 1.2217 & 1.769E+00 & JCMTPX_J185318.8+011459 \\
JCMTPP_J185319.2+012653 & 283.3298 & 1.4479 & 1.654E+00 & JCMTPX_J185318.8+011459 \\
JCMTPP_J185311.4+011617 & 283.2976 & 1.2713 & 1.638E+00 & JCMTPX_J185318.8+011459 \\
JCMTPP_J185316.7+012629 & 283.3195 & 1.4415 & 1.546E+00 & JCMTPX_J185318.8+011459 \\
JCMTPP_J185323.8+011047 & 283.3491 & 1.1798 & 1.528E+00 & JCMTPX_J185318.8+011459 \\
JCMTPP_J185316.7+011638 & 283.3195 & 1.2771 & 1.292E+00 & JCMTPX_J185318.8+011459 \\
JCMTPP_J185305.3+011549 & 283.2722 & 1.2637 & 1.278E+00 & JCMTPX_J185318.8+011459 \\
JCMTPP_J185320.5+012043 & 283.3353 & 1.3453 & 1.212E+00 & JCMTPX_J185318.8+011459 \\
JCMTPP_J185318.2+011623 & 283.3257 & 1.2730 & 1.204E+00 & JCMTPX_J185318.8+011459 \\
JCMTPP_J185320.3+011921 & 283.3347 & 1.3226 & 1.091E+00 & JCMTPX_J185318.8+011459 \\
JCMTPP_J185325.9+011407 & 283.3580 & 1.2351 & 9.760E-01 & JCMTPX_J185318.8+011459 \\
JCMTPP_J185321.5+012657 & 283.3395 & 1.4491 & 9.413E-01 & JCMTPX_J185318.8+011459 \\
JCMTPP_J185313.2+011253 & 283.3051 & 1.2147 & 9.409E-01 & JCMTPX_J185318.8+011459 \\
JCMTPP_J185312.2+011318 & 283.3010 & 1.2217 & 8.789E-01 & JCMTPX_J185318.8+011459 \\
JCMTPP_J185323.1+011421 & 283.3463 & 1.2392 & 8.282E-01 & JCMTPX_J185318.8+011459 \\
JCMTPP_J185319.3+011003 & 283.3305 & 1.1675 & 8.077E-01 & JCMTPX_J185318.8+011459 \\
JCMTPP_J185313.4+012333 & 283.3058 & 1.3925 & 7.729E-01 & JCMTPX_J185318.8+011459 \\
JCMTPP_J185315.7+012329 & 283.3154 & 1.3914 & 7.310E-01 & JCMTPX_J185318.8+011459 \\
JCMTPP_J185330.2+011243 & 283.3759 & 1.2118 & 7.295E-01 & JCMTPX_J185318.8+011459 \\
JCMTPP_J185305.0+011726 & 283.2708 & 1.2905 & 6.510E-01 & JCMTPX_J185318.8+011459\\
\enddata
\end{deluxetable}



\end{document}
